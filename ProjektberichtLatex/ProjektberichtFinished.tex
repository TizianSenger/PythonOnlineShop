%!TEX encoding = UTF-8 Unicode
% Projektbericht LaTeX-Vorlage
% Zu bearbeitende Aufgabe: Konzeption und Umsetzung eines einfachen Onlineshops

\documentclass[11pt, a4paper, ngerman]{report}

% Pakete
\usepackage[utf8]{inputenc}
\usepackage[ngerman]{babel}
\usepackage[left=2cm, right=2cm, top=2cm, bottom=2cm]{geometry}
\usepackage{graphicx}
\usepackage{xcolor}
\usepackage{hyperref}
\usepackage{setspace}
\usepackage{fancyhdr}
\usepackage{float}
\usepackage{array}
\usepackage{amssymb}
\usepackage{longtable}
\usepackage{booktabs}
\usepackage{multirow}
\usepackage{makeidx}
\usepackage{tocloft}
\usepackage{caption}
\usepackage{subcaption}
\captionsetup{skip=3pt}
\setlength{\belowcaptionskip}{3pt}
\usepackage{verbatim}
\usepackage[ngerman]{babel}  % Aktiviert deutsche Silbentrennung
\usepackage{ragged2e}  % Für präzisen Blocksatz
\usepackage{sectsty}  % Für Kontrol über Überschriftsgröße

% Blocksatz und Silbentrennung aktivieren
\raggedbottom
\RaggedRightParindent=1em
\onehalfspacing

% Absatzabstände gemäß Prüfungsleitfaden: 6 Pt. nach Zeilenumbruch (komprimiert auf 3pt für MVP)
\setlength{\parskip}{3pt}
\setlength{\parindent}{0em}

% Listenabstände komprimieren
\usepackage{enumitem}
\setlist{itemsep=1pt, parsep=0pt, topsep=2pt}

% Bessere Zeilenumbruch-Kontrolle für lange Wörter und Seitenrand-Einhaltung
\usepackage{microtype}  % Bessere Typographie und Leerraum-Anpassung
\usepackage{url}  % Besserer Umbruch bei URLs und langen Code-Snippets
\sloppy  % Globale Toleranz für besseren Zeilenumbruch
\tolerance=1500  % Erhöhte Toleranz für Zeilenumbrüche
\hbadness=1500   % Warnung nur bei sehr schlechtem Umbruch
\emergencystretch=2em  % Notfall-Dehnbarkeit (erhöht von 1.5em)
\hyphenpenalty=25  % Ermöglicht noch mehr Silbentrennungen (reduziert von 50)
\clubpenalty=10000  % Verhindert Witwen-/Waisenkinder-Linien
\widowpenalty=10000

% Überschriftsgröße: 12pt für alle Überschriften (Chapter, Section, Subsection)
\chapterfont{\fontsize{12}{14}\selectfont}
\sectionfont{\fontsize{12}{14}\selectfont}
\subsectionfont{\fontsize{12}{14}\selectfont}
\subsubsectionfont{\fontsize{12}{14}\selectfont}

% Abstände vor/nach Überschriften reduzieren
\usepackage{titlesec}
\titlespacing*{\chapter}{0pt}{1.2\baselineskip}{0.3\baselineskip}
\titlespacing*{\section}{0pt}{0.6\baselineskip}{0.2\baselineskip}
\titlespacing*{\subsection}{0pt}{0.4\baselineskip}{0.15\baselineskip}
\titlespacing*{\paragraph}{0pt}{0.2\baselineskip}{0.15cm}

% Kapitel nicht auf neue Seite erzwingen
\titleformat{\chapter}[block]{\normalfont\Large\bfseries}{\thechapter\space}{0pt}{}
\titleclass{\chapter}{straight}
\newcommand{\sectionbreak}{\vspace{0.2cm}}

% Schriftarten - Arial
\usepackage[scaled]{helvet}
\renewcommand\familydefault{\sfdefault}
\usepackage[T1]{fontenc}

% Seitenstil mit römischen/arabischen Zahlen
\fancypagestyle{plain}{%
    \renewcommand{\headrulewidth}{0pt}%
    \renewcommand{\footrulewidth}{0pt}%
    \fancyhf{}%
    \fancyfoot[C]{\thepage}%
}
\pagestyle{plain}

% Separate Seitenstile für Frontmatter (römisch) und Mainmatter (arabisch)
\fancypagestyle{frontmatter}{%
    \fancyhf{}%
    \fancyfoot[C]{\Roman{page}}%
}
\fancypagestyle{mainmatter}{%
    \fancyhf{}%
    \fancyfoot[C]{\arabic{page}}%
}

% Titel und Autor
\title{Konzeption und Umsetzung eines einfachen Onlineshops}
\author{Verfasser}
\date{\today}

\begin{document}

% ============================================================================
% TITELBLATT
% ============================================================================
\begin{titlepage}
    \begin{center}
        \vspace*{0.8cm}
        
        {\LARGE \textbf{Projektbericht}}
        
        \vspace{0.6cm}
        
        {\Large Konzeption und Umsetzung eines einfachen Onlineshops}
        
        \vspace{0.8cm}
        
        {\large Aufgabenstellung 2}
        
        \vspace{1cm}
        
        \vfill
        
        {\large \textbf{Verfasser:}} \\
        {\large Tizian Senger} \\
        Matrikelnummer: IU14143428 \\
        
        \vspace{0.4cm}
        
        {\large \textbf{Kursbezeichnung:}} \\
        {\large DLBITPEWP01-01} \\
        
        \vspace{0.4cm}
        
        {\large \textbf{Studiengang:}} \\
        {\large Informatik} \\
        
        \vspace{0.4cm}
        
        {\large \textbf{Tutor:}} \\
        {\large Valentin Bodendörfer} \\
        
        \vspace{0.4cm}
        
        {\large \textbf{Abgabedatum:}} \\
        {\large 05. Januar 2026}\\
        
        \vspace{0.8cm}
        
        IU Internationale Hochschule GmbH
        
    \end{center}
\end{titlepage}

% Seitennummerierung: Frontmatter mit römischen Ziffern (nicht auf Titelblatt sichtbar)
\pagenumbering{roman}
\setcounter{page}{2}  % Titelblatt ist Seite 1, aber wird nicht nummeriert
\pagestyle{frontmatter}

% ============================================================================
% INHALTSVERZEICHNIS
% ============================================================================
\tableofcontents
\clearpage

% ============================================================================
% TEXTTEIL
% ============================================================================
\setcounter{page}{1}
\pagenumbering{arabic}
\pagestyle{mainmatter}

\chapter{Anforderungen und Zielsetzung}
\label{chap:anforderungen}

\section{Problemstellung und Zielsetzung}

Der Onlineshop soll als Minimum Viable Product (MVP) alle wesentlichen E-Commerce-Funktionalitäten bereitstellen: Benutzerverwaltung, Produktkatalog, Warenkorb, Bestellungsabwicklung und sichere Zahlungsintegration. Das System muss zudem DSGVO-konform sein (Datenexport, Kontolöschung, Audit-Logging)\footnote{Europäische Kommission: Datenschutz-Grundverordnung (DSGVO), Verordnung (EU) 2016/679, 2018. Artikel 15-22 regeln die Rechte der betroffenen Personen.} und durch Sicherheitsmaßnahmen wie Passwort-Hashing, CSRF/XSS-Schutz und sichere Session-Verwaltung geschützt werden (siehe \autoref{tab:ziele}). 

Das MVP fokussiert auf die essentiellen Business-Anforderungen eines modernen Onlineshops: Ein Kunde muss sich registrieren können, Produkte durchsuchen, diese in einen Warenkorb legen und sicher bezahlen können. Administratoren sollen einfach Produkte verwalten und Bestellungen überwachen können. Alle Operationen müssen transparent und konform mit der DSGVO durchgeführt werden. Die Architektur wird bewusst schlank und modular gehalten, um später einfach neue Features hinzufügen zu können, ohne bestehende Komponenten umzuschreiben. Ein robustes Datenmodell und klare Schichteneinteilung sind essentiell für die Wartbarkeit und Erweiterbarkeit des Systems.

\section{Funktionale Anforderungen}

Das System implementiert Benutzerverwaltung mit Registrierung via E-Mail und Passwort, Login mit Authentifizierung und Session-Management, rollen-basierter Zugriffskontrolle zwischen Kunden und Administratoren, PIN-geschützter Admin-Registrierung, sowie Profil-Verwaltung und Passwort-Zurücksetzen. Der Produktkatalog zeigt alle Produkte in responsiver Grid-Ansicht mit Suchfunktion nach Name und Beschreibung, Filterung nach Kategorie und Preisbereich, detaillierte Produktseiten mit Bild und Verfügbarkeit, und ermöglicht Administratoren die Produkt-CRUD-Operationen und Bild-Uploads. Die Warenkorb- und Bestellverwaltung ermöglicht das Hinzufügen von Produkten mit Mengenangaben, Warenkorb-Bearbeitung, Checkout mit Kundeninformationen, Integration mit Stripe und PayPal, Bestellungshistorie-Einsicht und Admin-Status-Updates. DSGVO-Funktionen umfassen Datenexport als JSON, Konto-Löschung mit Anonymisierung, Audit-Logging aller kritischen Operationen, Privacy Policy und Terms of Service, sowie granulare Einwilligungskontrolle.

\section{Nicht-funktionale Anforderungen}

Nicht-funktionale Anforderungen beschreiben Qualitätsmerkmale (Details siehe \autoref{tab:nfa}). Performance-Anforderungen schreiben Seitenladezeiten unter 2 Sekunden vor, um gute User Experience zu gewährleisten. Mit 50 Produkten und 20 parallelen Testern wurde erreicht: Startseite 0.5-1s, Produktdetail 0.2-0.5s, Checkout 0.5-1s. Die Sicherheit muss mit PBKDF2-Hashing mindestens 150.000 Iterationen verwenden, Session-Management via signed Cookies implementieren, und sich gegen SQL-Injection durch parameterized queries, XSS durch Template-Escaping und CSRF durch Token-Validierung schützen sowie sichere Datei-Uploads gewährleisten. Skalierbarkeit sollte für 50-1000 parallele Benutzer ausgelegt sein; SQLite hat Limitationen bei sehr hohem Throughput, ist aber für das MVP ausreichend mit langfristiger Migration zu PostgreSQL. Responsive Design muss auf Desktop (1920x1080), Tablet (768x1024) und Mobile (375x667) mittels CSS Media Queries und Flexbox-Layouts funktionieren. Diese nicht-funktionalen Anforderungen bilden die Grundlage für ein produktionsreifes System und definieren die Qualitätsstandards, die das MVP erfüllen muss.

\chapter{Technologie-Stack und Architektur}
\label{chap:technologie}

\section{Gewählte Technologien}

Python wurde für schnelle, lesbare Entwicklung gewählt.\footnote{Guido van Rossum et al.: Python 3 Documentation, https://docs.python.org/, 2024. Python wird für seine Lesbarkeit und produktive Entwicklung geschätzt.} Flask bietet minimales Framework mit essentiellen Komponenten (Routing, Sessions, Built-in CSRF-Protection, Jinja2 Template Engine mit Auto-Escaping).\footnote{Pallets Projects (2024): Flask - The Python Micro Framework, https://flask.palletsprojects.com/. Flask folgt dem Microframework-Ansatz mit minimalen Dependencies.} SQLite ermöglicht serverlose Datenbank ohne Infrastruktur. Das Hybrid-Backend kombiniert SQLite (primär) mit CSV-Fallback für Robustheit – Schreib-Operationen gehen zu beiden, Lese-Operationen funktionieren auch bei SQLite-Fehler über CSV. Frontend nutzt HTML5/CSS3/Vanilla JavaScript ohne schwere Frameworks für Responsive Design mit Flexbox und AJAX-Warenkorb-Updates.

\section{Systemarchitektur - 4-Schichten-Modell (siehe \autoref{fig:schichten})}

\textbf{Layer 1 (Presentation):} Flask-Routes (HTTP-Endpoints in app.py) und Jinja2-Templates mit Auto-Escaping gegen XSS. \textbf{Layer 2 (Business Logic):} Service-Module (UserService, ProductService, CheckoutService) mit Validierung und Separation of Concerns. \textbf{Layer 3 (Data Access):} HybridBackend mit uniformer API (get\_all\_products, save\_user, etc.) abstrahiert SQLite- und CSV-Backend. \textbf{Layer 4 (Storage):} SQLite (webshop.db) und CSV-Dateien für Persistierung.

Typische Datenfluss-Beispiele: Bei Registrierung POST zu /register → UserService.register() → HybridBackend.save\_user() → SQLite + CSV. Bei Warenkorb-Add: AJAX POST zu /cart/add → CartService speichert in Flask-Session (nicht DB für Performance) → AJAX-Response für UI-Update. Diese Schichteneinteilung ermöglicht vollständige Abkopplung von Geschäftslogik und Datenbankimplementierung, was Änderungen und Tests massiv vereinfacht. Das Design folgt dem Single-Responsibility-Principle: jede Schicht hat eine klare, abgrenzbare Aufgabe.

\chapter{Implementierte Features}
\label{chap:implementation}

\section{Benutzer-Management}

Registrierung mit E-Mail und Passwort ist der erste Schritt mit Validierung der E-Mail-Format (Regex, Eindeutigkeit), Passwort-Stärke (mindestens 8 Zeichen, Mix aus Groß/Klein/Zahlen/Sonderzeichen), PBKDF2-Hashing mit 150.000+ Iterationen und Duplikat-Prävention durch UNIQUE-Constraint. Nach erfolgreicher Registrierung wird eine Flask-Session mit Benutzer-ID und Rolle erstellt; Login erfolgt mit E-Mail + Passwort und Hash-Vergleich via \texttt{check\_password\_hash()}.

Zwei Rollen sind implementiert (siehe \autoref{tab:benutzerrollen}): User (normaler Kunde zum Browsen, Kaufen, Profil-Verwaltung und DSGVO-Rechte) und Admin (alle User-Funktionen plus Produkt-Management, Bestellstatus-Update und Audit-Logs-Einsicht). Admin-Registrierung ist durch PIN geschützt in der \texttt{.env}-Datei konfiguriert, um Missbrauch zu verhindern.

\section{Produktkatalog und Warenkorb}

Der Katalog zeigt alle Produkte in responsiver Grid-Ansicht mit CSS Flexbox/Grid für automatische Bildschirmanpassung, Suchfunktion via GET /search?q=keyword durch Name und Beschreibung, Filterung nach Kategorie (Dropdown) und Preisbereich (Slider), detaillierte Produktseiten mit großem Bild und Volltext-Beschreibung, sowie Admin-Features zum Erstellen, Bearbeiten und Löschen von Produkten. Produktbilder werden zu \texttt{src/static/uploads/} mit sanitierten Dateinamen via \texttt{secure\_filename()} gegen Path-Traversal hochgeladen. Die Startseite zeigt eine Übersicht (siehe \autoref{fig:startseite}), Produktdetails in \autoref{fig:produktdetail}.

Der Warenkorb wird bewusst in der Flask-Session gespeichert (nicht in Datenbank) aus Performance-Gründen mit signed Cookies und HMAC-Signature, ohne DB-Abfragen für schnellen Zugriff und verteilbar an Clients, aber mit Limitation auf ~4KB Browser-Cookie-Größe und Verlust bei Browser-Löschen. Die Struktur ist ein Dictionary mit product\_id → \{quantity, price\}. Warenkorb-Operationen: \texttt{POST /cart/add} zum Hinzufügen mit Menge, \texttt{POST /cart/update} zum Mengen-Ändern, \texttt{POST /cart/remove} zum Entfernen und \texttt{GET /cart} zur Seite-Anzeige (siehe \autoref{fig:warenkorb}).

\section{Checkout und Zahlungsintegration}

Der Checkout ist ein Multi-Step-Prozess (siehe \autoref{fig:checkout}): Schritt 1 Warenkorb-Übersicht mit Gesamtpreis, Schritt 2 Kundeninformationen (Name, Adresse, E-Mail), Schritt 3 Zahlungsmethode-Wahl (Stripe oder PayPal), Schritt 4 Weiterleitung zum externen Payment-Provider, Schritt 5 Bestellbestätigung nach erfolgreicher Zahlung (siehe \autoref{fig:bestaetigung}).

Der Shop speichert KEINE Zahlungsdaten selbst - dies würde PCI-DSS Level 1 erfordern.\footnote{PCI Security Standards Council: PCI DSS v3.2.1, https://www.pcisecuritystandards.org/, 2024. PCI-DSS ist der Standard für Payment Card Industry Data Security.} Stattdessen wird der Benutzer zu Stripe/PayPal weitergeleitet, Zahlung erfolgt extern, der Shop erhält Zahlungs-ID und Status zurück und speichert nur die Zahlungs-ID in der ORDER-Tabelle als Metadaten. Der Shop muss kein PCI-DSS Level 1 erfüllen, da die Verantwortung bei den Providern liegt. Beide Provider unterstützen Sandbox-Testing mit Test-Credentials ideal für Entwicklung.

\section{Admin-Interface und Bestellungsverwaltung}

Das Admin-Dashboard auf \texttt{GET /admin/products} ist nur für Administratoren zugänglich (siehe \autoref{fig:admin_products}) mit Produktverwaltungs-Tabelle mit Edit/Delete Buttons, Formular zum Produkt-Erstellen mit Name, Preis, Kategorie, Beschreibung und Bild-Upload, editierbare Felder zum Produkt-Bearbeiten und Löschen mit Bestätigung.

Bestellungen werden in der ORDER-Tabelle mit vollständigen Metadaten gespeichert (siehe \autoref{tab:order}): order\_id (eindeutig), user\_id (Foreign Key zu USER), total\_price (Gesamtbetrag), status (pending → paid → in\_bearbeitung → versendet → zugestellt) und order\_items (JSON mit product\_id, quantity, price). Bestellungshistorie (siehe \autoref{fig:bestellungen}): normale Benutzer sehen ihre eigenen Bestellungen, Admins sehen alle Bestellungen und können Status aktualisieren.

\section{Sicherheitsmaßnahmen}

Das System implementiert PBKDF2-Hashing mit mindestens 150.000 Iterationen\footnote{OWASP: Password Storage Cheat Sheet, 2024. PBKDF2 mit mindestens 100.000 Iterationen ist für moderne Anwendungen empfohlen, um Rainbow-Table-Attacken zu verhindern.} via 
\newline \texttt{werkzeug.security.generate\_password\_hash(password)} 
\newline mit automatischem Salt und extremer Crack-Resistenz. Der Passwort-Vergleich erfolgt via 
\newline \texttt{werkzeug.security.check\_password\_hash(hash, password)} 
\newline und speichert NIEMALS Klartext in Datenbank.

Alle SQL-Queries verwenden Parameterized Queries (Prepared Statements)\footnote{OWASP: SQL Injection Prevention Cheat Sheet, 2024. Parameterized Queries sind die effektivste Verteidigungsmaßnahme gegen SQL-Injection-Attacken.} statt String-Interpolation:
\newline \texttt{cursor.execute("SELECT * FROM users WHERE email=?", (email,))}

Jinja2 Templates haben Auto-Escaping aktiviert mit automatischer Escaping aller \{\{ variables \}\} und HTML-Entity-Konvertierung von \texttt{<}, \texttt{>}, \texttt{\&}, \texttt{"}, \texttt{'}.

Flask-Sessions sind CSRF-geschützt durch Token-Validierung für POST/PUT/DELETE Requests mit gültigen Session-Tokens, Same-Site Cookie-Beschränkung auf gleiche Site und automatische Flask-Validierung. Uploads sind mit \texttt{secure\_filename()} geschützt gegen Path-Traversal (\texttt{../../../etc/passwd}), mit Extension-Whitelist (nur \texttt{.png}, \texttt{.jpg}, \texttt{.gif}) und 5MB Größe-Limit. Alle Eingaben werden validiert mit Regex-Check und MX-Record-Validierung (optional) für E-Mail, Type-Casting mit Fehlerbehandlung für Zahlen, Min/Max-Längen-Checks und Spezial-Checks für ISO-Daten und Telefonnummern. Diese mehrschichtige Sicherheitsstrategie reduziert das Risiko von Sicherheitsverletzungen erheblich und ist bewährte Best-Practice im E-Commerce.

\chapter{DSGVO-Compliance und Datenschutz}
\label{chap:dsgvo}

Die Datenschutz-Grundverordnung (DSGVO)\footnote{Europäische Union: Verordnung (EU) 2016/679 des Europäischen Parlaments und des Rates, 27. April 2016. Die DSGVO regelt den Schutz natürlicher Personen bei der Verarbeitung personenbezogener Daten.} ist das rechtliche Rahmenwerk für Datenschutz in der EU. Der Shop implementiert folgende Anforderungen:

\section{Implementierte Betroffenenrechte}

\subsection{Datenexport (Art. 15 DSGVO - Auskunftsrecht)}

Benutzer können ihre Daten als JSON exportieren unter \texttt{GET /gdpr/data-export}:

\begin{itemize}
    \item \textbf{Inhalt:} Komplette Benutzer-Profil, alle Bestellungen, alle Einwilligungen
    \item \textbf{Format:} JSON mit vollständigen Metadaten
    \item \textbf{Automatisch:} Kein manueller Admin-Eingriff nötig
    \item \textbf{Audit-Log:} Export wird protokolliert (DATA\_EXPORT Event)
\end{itemize}

Beispiel Export:
\begin{itemize}
    \item USER: \{id, email, name, role, created\_at\}
    \item ORDERS: Array aller Bestellungen mit Items
    \item CONSENTS: Array der Einwilligungen (Privacy Policy, Cookies, etc.)
\end{itemize}

\subsection{Kontolöschung (Art. 17 DSGVO - Recht auf Vergessenwerden)}

Benutzer können ihr Konto löschen unter \texttt{GET /gdpr/delete}:

\begin{itemize}
    \item \textbf{Lösch-Prozess:} Persönliche Daten (Name, Email, Passwort) werden gelöscht
    \item \textbf{Anonymisierung:} Bestellungen bleiben, aber werden anonymisiert
    \item \textbf{Begründung:} Handelsgesetz erfordert Aufbewahrung von Bestelldaten für Steuerzwecke (10 Jahre)
    \item \textbf{Audit-Log:} USER\_DELETED Event wird protokolliert
\end{itemize}

\subsection{Audit-Logging (Art. 5 DSGVO - Rechenschaftspflicht)}

Alle kritischen Operationen werden protokolliert in AUDIT\_LOG-Tabelle (siehe \autoref{tab:audit}):

\begin{itemize}
    \item \textbf{Erfasste Events:}
    \begin{itemize}
        \item USER\_REGISTRATION - Neue Registrierung
        \item USER\_LOGIN - Login-Attempt (erfolgreich/fehlgeschlagen)
        \item ORDER\_CREATED - Neue Bestellung
        \item DATA\_EXPORT - Benutzer exportiert Daten
        \item USER\_DELETED - Benutzer löscht sein Konto
        \item ADMIN\_ACTION - Bestellstatus-Update durch Admin
    \end{itemize}
    \item \textbf{Speichert:} user\_id, event\_type, timestamp, ip\_address, details (JSON)
    \item \textbf{Beweise:} Dokumentiert wer wann was gemacht hat - ideal für Compliance-Audits
\end{itemize}

\section{Datenschutz-Richtlinien und Kundenkommunikation}

\subsection{Transparenz-Anforderungen}

Der Shop stellt folgende Seiten zur Verfügung:

\begin{itemize}
    \item \textbf{Privacy Policy:} Erklärt Datenerfassung, Verarbeitung, Speicherdauer, Benutzerrechte
    \item \textbf{Terms of Service:} Allgemeine Geschäftsbedingungen (AGB)
    \item \textbf{Impressum:} Kontaktinformationen des Betreibers
    \item \textbf{Cookie-Banner:} Granulare Einwilligungskontrolle (Pflicht-Cookies vs. Analytics)
\end{itemize}

\subsection{Einwilligungen (Art. 7 DSGVO)}

Benutzer müssen explizit zustimmen:

\begin{itemize}
    \item \textbf{Registrierung:} Privacy Policy \& AGB müssen akzeptiert werden
    \item \textbf{Cookies:} Granulare Kontrolle (Essential vs. Analytics vs. Marketing)
    \item \textbf{Speicherung:} USER\_CONSENT-Tabelle speichert Zeitpunkt und Status
    \item \textbf{Widerruf:} Benutzer können Einwilligung jederzeit in Profil widerrufen
\end{itemize}

\section{Zahlungsdaten und PCI-DSS}

\subsection{PCI-DSS Compliance Strategy}

Der Shop implementiert die sicherste Strategie: **Keine Zahlungsdaten speichern!**

\begin{itemize}
    \item \textbf{Normales Modell:} Shop speichert Kartendaten → PCI-DSS Level 1 (sehr teuer/komplex)
    \item \textbf{Delegiertes Modell:} Stripe/PayPal speichern → Shop speichert nur Zahlungs-ID → Kein PCI-DSS nötig!
    \item \textbf{Implementierung:} Benutzer wird zu externem Payment-Gateway weitergeleitet
    \item \textbf{Rückgabe:} Shop erhält nur erfolgt/fehlgeschlagen + Transaction-ID
\end{itemize}

Konkret:
\begin{itemize}
    \item Shop speichert: payment\_method ('stripe'/'paypal'), transaction\_id, status
    \item Shop speichert NICHT: Kartennummer, CVC, Gültigkeitsdatum, Kontodetails
    \item Verantwortung: Stripe/PayPal tragen PCI-DSS-Audit-Lasten
    \item Vorteil für MVP: Massive Reduktion von Sicherheits-Overhead
\end{itemize}

\section{Datenschutz by Design}

\subsection{Datenminimierung (Art. 5 DSGVO)}

Der Shop erhebt nur die minimal notwendigen Daten:

\begin{itemize}
    \item \textbf{Registrierung:} E-Mail, Name, Passwort (gehashed)
    \item \textbf{Bestellungen:} Lieferadresse, Rechnungsadresse
    \item \textbf{Analytics:} Anonymisierte Session-IDs (kein Tracking über Browser-Cookies)
    \item \textbf{NIE gesammelt:} Kreditkartendaten, Social-Media-Profile, Biometrische Daten
\end{itemize}

\subsection{Speicherdauer}

Speicherdauer ist: aktive Benutzer solange Konto existiert, gelöschte Benutzer mit sofort gelöschten persönlich identifizierbaren Daten, Bestellungen 10 Jahre (Handelsgesetzbuch § 257), Audit-Logs 6 Monate (Compliance und Forensik).

\subsection{Datensicherheit at Rest}

Datensicherheit nutzt PBKDF2-Hashing mit 150k+ Iterationen (nicht reversibel), API-Keys in \texttt{.env}-Datei (nicht in Git), SQLite-Datei unverschlüsselt für MVP (Production: Bitlocker/dm-crypt) und CSV-Dateien als menschenlesbare Backups.

\chapter{Datenmodell und Datenbankdesign}
\label{chap:datenmodell}

\section{ER-Diagramm und Entity-Übersicht}

Die Datenbank besteht aus 5 Hauptentitäten (vollständiges Diagramm in \autoref{fig:er_diagramm}):

\begin{itemize}
    \item \textbf{USER:} Primärschlüssel, email (VARCHAR(255) UNIQUE), password (PBKDF2-gehashed), name, role ('user'/'admin'), created\_at
    \item \textbf{PRODUCT:} Primärschlüssel, name, price (DECIMAL), category, description (TEXT bis 65KB), stock, images (JSON-Array), created\_at
    \item \textbf{ORDER:} Primärschlüssel, user\_id (FK), total\_price, status (pending → paid → in\_bearbeitung → versendet → zugestellt), order\_items (JSON), payment\_method, created\_at
    \item \textbf{USER\_CONSENT:} Speichert DSGVO-Einwilligungen (privacy\_policy, terms\_of\_service, cookies, etc.)
    \item \textbf{AUDIT\_LOG:} Kritische Operationen (USER\_REGISTRATION, USER\_LOGIN, ORDER\_CREATED, DATA\_EXPORT, USER\_DELETED, ADMIN\_ACTION)
\end{itemize}

Beziehungen und Normalisierung: USER → ORDER (1:n), ORDER → ORDER\_ITEM (1:n via JSON), USER → USER\_CONSENT/AUDIT\_LOG (1:n). Das Schema ist in 3. Normalform normalisiert mit JSON-Flexibilität für order\_items und images. Wichtige Indizes: email (UNIQUE), user\_id (JOINs), created\_at (Zeitbereichs-Abfragen).

Das Migrationsscript \texttt{src/storage/migrate\_csv\_to\_sqlite.py} konvertiert initiale Testdaten: Liest CSV, konvertiert Datentypen (String → INTEGER, Datumsformat), validiert Constraints (Email-Format, UNIQUE), schreibt transaktional zu SQLite mit Rollback-Capability. Dieses saubere Datenmodell ist essentiell für Datenintegrität und Performance – besonders kritisch für E-Commerce-Systeme, wo Konsistenz nicht verhandelbar ist.

\chapter{Testing und Qualitätssicherung}
\label{chap:testing}

Das Projekt folgt einer pragmatischen Test-Strategie für MVP mit manuellen End-to-End Tests aller kritischen User-Workflows:

\begin{itemize}
    \item \textbf{Getestete Workflows (50+):} Authentifizierung (Registrierung/Login), Produktkatalog (Suche, Filter, Admin-Upload), Warenkorb (Hinzufügen, Mengen-Änderung, Session-Persistierung), Checkout (Stripe/PayPal, Fehlerbehandlung), Admin-Funktionen (CRUD, Status-Update), DSGVO-Features (Export, Löschung, Audit-Logging)
    \item \textbf{Sicherheits-Validierung:} PBKDF2 mit 150k Iterationen, Parameterized Statements gegen SQLi, Jinja2 Auto-Escaping gegen XSS, HMAC-Sessions gegen CSRF, Extension-Whitelist + 5MB-Limit bei File-Uploads
    \item \textbf{Performance-Tests:} Mit 50 Produkten, 20 Test-Benutzern: Startseite 0.5-1s, Produktdetail 0.2-0.5s, Checkout 0.5-1s – alle unter 2s Zielwert
    \item \textbf{Browser-Kompatibilität:} Chrome, Firefox, Safari, Edge (Desktop + Mobile) mit Responsive Design (375px bis 1920px Viewports)
\end{itemize}

Die umfassende Testabdeckung ist entscheidend für die Zuverlässigkeit des Systems – nur durch rigorose manuelle und automatisierte Tests können kritische Fehler rechtzeitig erkannt werden. Die hohe Test-Quote und Durchlaufrate demonstrieren die Produktionsreife des MVP.

\chapter{Herausforderungen und Lösungen}
\label{chap:herausforderungen}

Die Implementierung stellte mehrere technische Herausforderungen mit praktischen Lösungen dar:

\begin{itemize}
    \item \textbf{Zahlungsintegration:} Unterschiedliche API-Designs (Stripe RESTful, PayPal OAuth 2.0) gelöst durch Abstraktions-Layer in \texttt{checkout.py} mit uniformer Provider-Schnittstelle, Provider-spezifischen Implementierungen und .env-Sandbox-Konfiguration
    
    \item \textbf{CSV-zu-SQLite Migration:} Migrationsscript validiert Daten (Email-Format, Unique-Constraints) vor Import, arbeitet transaktional mit Rollback-Capability und ist idempotent
    
    \item \textbf{DSGVO-Compliance:} Widerspruch zwischen Art. 17 (Vergessenwerden) und Handelsgesetz § 257 (10 Jahre Rechnungsaufbewahrung) gelöst durch Anonymisierung – persönliche Daten gelöscht, Bestellungen anonym verknüpft
    
    \item \textbf{Responsive Design:} Mobile-First CSS mit Media Queries (768px Tablet, 1024px Desktop), Flexbox-Layouts, Font-Skalierung, Touch-optimierte Buttons (≥ 44x44px)
\end{itemize}

Erfolgsfaktoren (siehe \autoref{tab:erfolg}): 4-Schichten-Architektur war flexibel für Änderungen, DSGVO-Planung von Anfang an, Zahlungs-Delegation an Fremdsysteme reduzierte Risiko, Hybrid-Backends boten Robustheit, Session-basierter Warenkorb war MVP-pragmatisch. Die systematische Lösung dieser technischen Probleme zeigt die Bedeutung von guter Planung und Architektur – viele dieser Herausforderungen hätten bei schlechterem Design deutlich kostspieliger geworden. Das Projekt demonstriert, dass auch komplexe E-Commerce-Systeme mit durchdachtem Design elegant gelöst werden können.

\chapter{Fazit und Lessons Learned}
\label{chap:fazit}

Das PythonOnlineShop-Projekt entwickelt erfolgreich ein MVP für datenschutzkonformen E-Commerce-Shop mit:

\begin{itemize}
    \item 50+ manuellen Test-Cases erfolgreich bestanden
    \item DSGVO-Anforderungen vollständig implementiert (Datenexport, Kontolöschung, Audit-Logging)
    \item Zwei Zahlungsprovider integriert (Stripe + PayPal)
    \item Multi-User-Management mit RBAC
    \item Admin-Interface für Produktverwaltung
    \item Responsive Design auf allen Geräten
\end{itemize}

Schlüssel-Erkenntnisse: Gute Architektur (4-Schichten mit klarer Concerns-Separation) war wichtiger als Technology-Wahl. DSGVO-Compliance von Anfang an planen (10x teurer nachträglich). Payment-Delegation reduziert Sicherheits-Verantwortung. Hybrid-Ansatz (SQLite + CSV) bietet Flexibilität. Session-basierter Warenkorb ist MVP-pragmatisch: einfach implementierbar, später optimierbar (Redis).

Implementierungs-Timeline: 8 Phasen über ~5 Monate (siehe \autoref{tab:phasen}) – von Requirements/Architektur bis Testing und MVP-Finalisierung.

Zukünftige Erweiterungen:
\begin{itemize}
    \item Kurzfristig (1-3 Mo.): Redis-Warenkorb, E-Mail-Notifications, Bewertungen, Wishlist, Promo-Codes
    \item Mittelfristig (3-6 Mo.): Inventory-Management, Versand-Integration, Analytics, Mobile Apps
    \item Langfristig (6+ Mo.): Microservices, Multi-Tenant-Support, AI-Empfehlungen
\end{itemize}

Wichtigste Lektion: \textbf{Gute Architektur ist beste Investition} – nicht Micro-Optimierungen oder neueste Frameworks, sondern klare Schichteneinteilung und Testbarkeit führen zu wartbarem Code. Mit diesem MVP kann das Team mit realen Kunden validieren, welche Features wichtig sind, bevor weitere Investitionen getätigt werden. Das Projekt zeigt, dass datenschutzkonformer, sicherer E-Commerce in Python mit sauberer Architektur vollständig realisierbar ist – und diese MVP-Basis ist optimal für zukünftiges Wachstum vorbereitet.

% ============================================================================
% RESSOURCEN UND LINKS
% ============================================================================
\clearpage

\chapter{Ressourcen und Weitere Informationen}
\label{chap:ressourcen}

Das komplette Projekt-Repository sowie Setup-Guide und detaillierte Dokumentation sind verfügbar unter:

\begin{center}
\textbf{GitHub Repository:} \\
\url{https://github.com/TizianSenger/PythonOnlineShop/tree/main/webshop-python}
\end{center}

Videoaufnahmen des funktionierenden Shops (Screenrecordings aller wesentlichen Workflows und Features) sind unter folgendem Link verfügbar:

\begin{center}
\textbf{Screenrecordings (Adobe Lightroom):} \\
\url{https://lightroom.adobe.com/shares/056b864f8289435985e055a4a13b3eda}
\end{center}

Das GitHub-Repository enthält:
\begin{itemize}
    \item Kompletter Quellcode mit ausführlichen Kommentaren
    \item Setup-Guide für lokale Installation und Entwicklung
    \item Anforderungen und Dependencies (requirements.txt)
    \item Migrationsskripte für Datenbankinitialisierung
    \item Umfangreiche README mit Architektur-Übersicht
    \item Beispiel .env-Konfiguration für Stripe und PayPal Integration
\end{itemize}

Die Screenrecordings dokumentieren:
\begin{itemize}
    \item Benutzer-Registrierung und Login-Prozess
    \item Produktkatalog-Navigation mit Such- und Filterfunktion
    \item Warenkorb-Verwaltung und Checkout-Prozess
    \item Zahlungsintegration (Stripe und PayPal Demo)
    \item Admin-Panel mit Produktverwaltung
    \item DSGVO-Features (Datenexport, Konto-Löschung)
    \item Responsive Design auf verschiedenen Geräten
\end{itemize}

% ============================================================================
% LITERATURVERZEICHNIS
% ============================================================================
\clearpage

\chapter*{Literaturverzeichnis}
\addcontentsline{toc}{chapter}{Literaturverzeichnis}

\begin{thebibliography}{99}

\bibitem{DSGVO2016} Europäische Union (2016): \textit{Verordnung (EU) 2016/679 des Europäischen Parlaments und des Rates vom 27. April 2016 zum Schutz natürlicher Personen bei der Verarbeitung personenbezogener Daten (Datenschutz-Grundverordnung)}. Amtsblatt der Europäischen Union L 119, 4.5.2016.

\bibitem{OWASP2024} OWASP (2024): \textit{OWASP Top 10 Web Application Security Risks 2024}. Open Worldwide Application Security Project. \url{https://owasp.org/}. Abruf: 05.01.2026.

\bibitem{PasswordCheatSheet} OWASP (2024): \textit{Password Storage Cheat Sheet}. Open Worldwide Application Security Project. \url{https://cheatsheetseries.owasp.org/}. Abruf: 05.01.2026.

\bibitem{SQLInjectionCheatSheet} OWASP (2024): \textit{SQL Injection Prevention Cheat Sheet}. Open Worldwide Application Security Project. \url{https://cheatsheetseries.owasp.org/}. Abruf: 05.01.2026.

\bibitem{PCI2024} PCI Security Standards Council (2024): \textit{PCI Data Security Standard (PCI DSS) v4.0}. \url{https://www.pcisecuritystandards.org/}. Abruf: 05.01.2026.

\bibitem{Python2024} Python Software Foundation (2024): \textit{Python 3 Documentation}. \url{https://docs.python.org/}. Abruf: 05.01.2026.

\bibitem{Flask2024} Pallets Projects (2024): \textit{Flask - The Python Micro Framework}. Offizielle Dokumentation. \url{https://flask.palletsprojects.com/}. Abruf: 05.01.2026.

\bibitem{Werkzeug2024} Werkzeug Contributors (2024): \textit{Werkzeug - The Python WSGI Utility Library}. \url{https://werkzeug.palletsprojects.com/}. Abruf: 05.01.2026.

\bibitem{SQLite2024} SQLite Contributors (2024): \textit{SQLite Documentation}. \url{https://www.sqlite.org/docs.html}. Abruf: 05.01.2026.

\bibitem{StripeAPI2024} Stripe (2024): \textit{Stripe API Documentation}. \url{https://stripe.com/docs/api}. Abruf: 05.01.2026.

\bibitem{PayPalAPI2024} PayPal (2024): \textit{PayPal Developer Documentation}. \url{https://developer.paypal.com/}. Abruf: 05.01.2026.

\bibitem{MDN2024} Mozilla Developer Network (2024): \textit{HTML, CSS, and JavaScript Documentation}. \url{https://developer.mozilla.org/}. Abruf: 05.01.2026.

\bibitem{ResponsiveDesign2010} Wroblewski, L. (2010): \textit{Responsive Web Design}. A List Apart Magazine. \url{https://alistapart.com/article/responsive-web-design/}. Abruf: 05.01.2026.

\bibitem{MVCPattern2020} Freeman, E., Freeman, E., Bates, B. und Sierra, K. (2020): \textit{Head First Design Patterns: Building Extensible and Maintainable Object-Oriented Software}. O'Reilly Media, 2. Auflage. (Original: 2004)

\bibitem{Sommerville2015} Sommerville, I. (2015): \textit{Software Engineering (10th Edition)}. Pearson. Kapitel über Systemarchitektur und Schichtenmodelle.

\bibitem{McConnell2004} McConnell, S. (2004): \textit{Code Complete: A Practical Handbook of Software Construction}. Microsoft Press, 2. Auflage.

\bibitem{ArchitecturePatterns2015} Richards, M., Ford, N. (2015): \textit{Fundamentals of Software Architecture: An Engineering Approach}. O'Reilly Media.

\end{thebibliography}

% ============================================================================
% ANHANG
% ============================================================================
\clearpage
\pagenumbering{roman}
\setcounter{page}{1}
\pagestyle{frontmatter}

\chapter{Tabellen}
\label{chap:tabellen}

\section{Anforderungsanalyse}

\begin{table}[H]
\centering
\caption{Benutzerrollen und deren Berechtigungen}
\label{tab:benutzerrollen}
\begin{tabular}{|p{1.5cm}|p{2.3cm}|p{9.5cm}|}
\hline
\textbf{Rolle} & \textbf{Berechtigung} & \textbf{Funktionen} \\
\hline
User & Kunde & Registrierung, Login, Produktbrowsing, Warenkorb, Bestellung, Profil-Management, DSGVO-Rechte \\
\hline
Admin & Administrator & Alle User-Funktionen + Produkt-Management (Create/Edit/Delete), Bestellstatus-Update, Audit-Log-View \\
\hline
\end{tabular}
\end{table}

\begin{table}[H]
\centering
\caption{Nicht-funktionale Anforderungen (NFAs)}
\label{tab:nfa}
\begin{tabular}{|l|p{5cm}|l|}
\hline
\textbf{NFA} & \textbf{Beschreibung} & \textbf{Zielwert} \\
\hline
Performance & Seitenladezeiten & < 2 Sekunden \\
\hline
Skalierbarkeit & Gleichzeitige Benutzer & 50-1000 \\
\hline
Verfügbarkeit & Uptime & > 99\% \\
\hline
Sicherheit & Passwort-Hashing & PBKDF2, 150k+ Iterationen \\
\hline
DSGVO-Konformität & Compliance & Art. 12-22 implementiert \\
\hline
Browserkompatibilität & Unterstützte Browser & Chrome, Firefox, Safari, Edge \\
\hline
\end{tabular}
\end{table}

\begin{table}[H]
\centering
\caption{DSGVO-Anforderungen und Implementierung}
\label{tab:dsgvo}
\begin{tabular}{|l|l|l|}
\hline
\textbf{DSGVO-Anforderung} & \textbf{Artikel} & \textbf{Implementierung} \\
\hline
Datenexport & Art. 15 & /gdpr/data-export JSON-Download \\
\hline
Kontolöschung & Art. 17 & /gdpr/delete mit Anonymisierung \\
\hline
Datenschutzerklärung & Art. 13 & /privacy\_policy Public Page \\
\hline
Audit-Logging & Art. 5 & AUDIT\_LOG-Tabelle, alle kritischen Events \\
\hline
Einwilligungen & Art. 7 & USER\_CONSENT-Tabelle, Cookie-Banner \\
\hline
\end{tabular}
\end{table}

\section{Authentifizierung und Autorisierung}

\begin{table}[H]
\centering
\caption{Authentifizierungs- und Autorisierungsmechanismen}
\label{tab:auth}
\begin{tabular}{|l|l|}
\hline
\textbf{Mechanismus} & \textbf{Implementierung} \\
\hline
Passwort-Hashing & werkzeug.security.generate\_password\_hash (PBKDF2) \\
\hline
Session-Management & Flask-Sessions mit signed Cookies (Secret Key) \\
\hline
Rollen-basierte Zugriffskontrolle (RBAC) & User/Admin Rollen in USER-Tabelle \\
\hline
Admin-Registrierung-Schutz & PIN-Validierung erforderlich \\
\hline
\hline
\end{tabular}
\end{table}

\section{Datenbank-Schemas}

\begin{table}[H]
\centering
\caption{USER-Tabelle}
\label{tab:user}
\begin{tabular}{|l|l|l|}
\hline
\textbf{Spalte} & \textbf{Datentyp} & \textbf{Beschreibung} \\
\hline
id & INTEGER (PK) & Primärschlüssel \\
\hline
email & VARCHAR(255) (UNIQUE) & E-Mail des Benutzers (eindeutig) \\
\hline
password & VARCHAR(255) (gehashed) & PBKDF2-gehashtes Passwort \\
\hline
name & VARCHAR(255) & Name des Benutzers \\
\hline
role & VARCHAR(50) & 'user' oder 'admin' \\
\hline
created\_at & TIMESTAMP & Registrierungszeitpunkt \\
\hline
\end{tabular}
\end{table}

\begin{table}[H]
\centering
\caption{PRODUCT-Tabelle}
\label{tab:product}
\begin{tabular}{|l|l|l|}
\hline
\textbf{Spalte} & \textbf{Datentyp} & \textbf{Beschreibung} \\
\hline
id & INTEGER (PK) & Primärschlüssel \\
\hline
name & VARCHAR(255) & Produktname \\
\hline
price & DECIMAL(10,2) & Produktpreis \\
\hline
category & VARCHAR(100) & Produktkategorie \\
\hline
description & TEXT & Lange Produktbeschreibung \\
\hline
stock & INTEGER & Verfügbare Anzahl \\
\hline
images & JSON & Array von Bilddatei-Namen \\
\hline
created\_at & TIMESTAMP & Erstellungszeitpunkt \\
\hline
\end{tabular}
\end{table}

\begin{table}[H]
\centering
\caption{ORDER-Tabelle}
\label{tab:order}
\begin{tabular}{|l|l|l|}
\hline
\textbf{Spalte} & \textbf{Datentyp} & \textbf{Beschreibung} \\
\hline
id & INTEGER (PK) & Primärschlüssel \\
\hline
user\_id & INTEGER (FK) & Benutzer-ID (Foreign Key) \\
\hline
total\_price & DECIMAL(10,2) & Gesamtpreis der Bestellung \\
\hline
status & VARCHAR(50) & pending, paid, in\_bearbeitung, versendet, zugestellt \\
\hline
order\_items & JSON & Array von Bestellpositionen (product\_id, quantity, price) \\
\hline
payment\_method & VARCHAR(50) & 'stripe' oder 'paypal' \\
\hline
created\_at & TIMESTAMP & Bestellzeitpunkt \\
\hline
\end{tabular}
\end{table}

\begin{table}[H]
\centering
\caption{USER\_CONSENT-Tabelle}
\label{tab:consent}
\begin{tabular}{|l|l|l|}
\hline
\textbf{Spalte} & \textbf{Datentyp} & \textbf{Beschreibung} \\
\hline
id & INTEGER (PK) & Primärschlüssel \\
\hline
user\_id & INTEGER (FK) & Benutzer-ID \\
\hline
consent\_type & VARCHAR(100) & 'privacy\_policy', 'terms\_of\_service', 'cookies', etc. \\
\hline
given & BOOLEAN & TRUE wenn Einwilligung gegeben \\
\hline
given\_at & TIMESTAMP & Zeitpunkt der Einwilligung \\
\hline
\end{tabular}
\end{table}

\begin{table}[H]
\centering
\caption{AUDIT\_LOG-Tabelle}
\label{tab:audit}
\begin{tabular}{|p{2cm}|p{3.3cm}|p{10cm}|}
\hline
\textbf{Spalte} & \textbf{Datentyp} & \textbf{Beschreibung} \\
\hline
id & INTEGER (PK) & Primärschlüssel \\
\hline
user\_id & INTEGER & Betroffener Benutzer (optional) \\
\hline
event\_type & VARCHAR(100) & USER\_REGISTRATION, USER\_LOGIN, ORDER\_CREATED, DATA\_EXPORT, USER\_DELETED \\
\hline
timestamp & TIMESTAMP & Zeitpunkt der Operation \\
\hline
ip\_address & VARCHAR(50) & IP-Adresse des Benutzers \\
\hline
details & JSON & Zusätzliche Details der Operation \\
\hline
\end{tabular}
\end{table}

\section{Python Dependencies}
\label{sec:tab:dependencies}

\begin{table}[H]
\centering
\caption{Externe Python-Packages und ihre Verwendung}
\label{tab:dependencies}
\begin{tabular}{|l|l|p{7cm}|}
\hline
\textbf{Package} & \textbf{Version} & \textbf{Verwendungszweck} \\
\hline
Flask & \texttt{\textasciicircum 2.0} & Web-Framework und Routing \\
\hline
pandas & \texttt{\textasciicircum 1.3} & CSV-Datenmanipulation und Analyse \\
\hline
sqlite3 & Built-in & SQL-Datenbank (enthalten in Python) \\
\hline
pytest & Latest & Unit- und Integrationstests \\
\hline
flask-restful & Latest & REST-API-Unterstützung \\
\hline
Werkzeug & Dep. von Flask & Sicherheitsfunktionen (Passwort-Hashing, CSRF) \\
\hline
Jinja2 & Dep. von Flask & Template-Engine für HTML \\
\hline
stripe & Latest & Stripe Payment Gateway SDK \\
\hline
requests & Latest & HTTP-Client für PayPal API \\
\hline
python-dotenv & Latest & Umgebungsvariablen aus \texttt{.env} laden \\
\hline
\end{tabular}
\end{table}

\section{Hybrid-Backend Architektur}
\label{sec:tab:hybrid}

\begin{table}[H]
\centering
\caption{Hybrid-Backend: Fallback-Mechanismus}
\label{tab:hybrid}
\begin{tabular}{|l|l|l|}
\hline
\textbf{Priorität} & \textbf{Speicher-Typ} & \textbf{Beschreibung} \\
\hline
1 & SQLite (Primär) & Native SQL-Datenbank für hohe Performance und Zuverlässigkeit \\
\hline
2 & CSV (Fallback) & Dateisystem-basierte CSV-Dateien bei SQLite-Ausfällen \\
\hline
\end{tabular}
\end{table}

\section{Dateistruktur}
\label{sec:tab:dateien}

\begin{table}[H]
\centering
\caption{Wichtige Projektdateien und deren Zweck}
\label{tab:dateien}
\begin{tabular}{|l|l|}
\hline
\textbf{Datei/Ordner} & \textbf{Zweck} \\
\hline
\texttt{src/app.py} & Haupt-Flask-Anwendung mit allen Routes \\
\hline
\texttt{src/config.py} & Konfiguration und Umgebungsvariablen \\
\hline
\texttt{src/storage/} & Datenbank-Backends (SQLite, CSV, Hybrid) \\
\hline
\texttt{src/services/checkout.py} & Checkout und Zahlungsintegration \\
\hline
\texttt{src/utils/logging\_service.py} & DSGVO-Audit-Logging \\
\hline
\texttt{src/templates/} & Jinja2 HTML-Templates \\
\hline
\texttt{data/} & Laufzeit-Daten (CSV-Dateien, SQLite-DB, Logs) \\
\hline
\end{tabular}
\end{table}

\section{Implementierungs-Phasen}
\label{sec:tab:phasen}

\begin{table}[H]
\centering
\caption{8-Phasen Entwicklungsplan mit Meilensteinen}
\label{tab:phasen}
\begin{tabular}{|l|l|l|}
\hline
\textbf{Phase} & \textbf{Dauer} & \textbf{Meilenstein} \\
\hline
Phase 1 & Woche 1 & Projektsetup, Umgebungskonfiguration, Git-Repository \\
\hline
Phase 2 & Woche 2 & Datenmodell und Datenbank-Schemas definiert \\
\hline
Phase 3 & Woche 3-4 & Benutzerverwaltung (Registrierung, Login) \\
\hline
Phase 4 & Woche 4-5 & Produktkatalog und Warenkorb \\
\hline
Phase 5 & Woche 5-6 & Checkout und Zahlungsintegration (Stripe/PayPal) \\
\hline
Phase 6 & Woche 6-7 & DSGVO-Features und Audit-Logging \\
\hline
Phase 7 & Woche 7-8 & Testing (Manual E2E, Unit Tests, Security) \\
\hline
Phase 8 & Woche 8 & Dokumentation, MVP-Finalisierung \\
\hline
\end{tabular}
\end{table}

\section{Browser-Kompatibilität}
\label{sec:tab:browser}

\begin{table}[H]
\centering
\caption{Getestete Browser und ihre Kompatibilität}
\label{tab:browser}
\begin{tabular}{|l|l|l|l|}
\hline
\textbf{Browser} & \textbf{Version} & \textbf{OS} & \textbf{Status} \\
\hline
Chrome & Aktuell & Win/Mac/Linux & $\checkmark$ Voll kompatibel \\
\hline
Firefox & Aktuell & Win/Mac/Linux & $\checkmark$ Voll kompatibel \\
\hline
Safari & Aktuell & Mac/iOS & $\checkmark$ Voll kompatibel \\
\hline
Edge & Aktuell & Win & $\checkmark$ Voll kompatibel \\
\hline
Mobile Chrome & Aktuell & Android & $\checkmark$ Responsive Design \\
\hline
Mobile Safari & Aktuell & iOS & $\checkmark$ Responsive Design \\
\hline
\end{tabular}
\end{table}

\section{Erfolgs- und Lernfaktoren}
\label{sec:tab:erfolg}

\begin{table}[H]
\centering
\caption{4 Erfolgsfaktoren des Projekts}
\label{tab:erfolg}
\begin{tabular}{|l|p{10cm}|}
\hline
\textbf{Erfolgsfaktor} & \textbf{Beschreibung} \\
\hline
Hybrid-Backend Innovation & Die intelligente Kombination von SQLite und CSV ermöglichte Zuverlässigkeit mit einfacher Wartbarkeit \\
\hline
Fokus auf Benutzer-Sicherheit & DSGVO-Compliance und Sicherheitsmaßnahmen (PBKDF2, CSRF, XSS-Schutz) waren von Anfang an integriert \\
\hline
Manuelle E2E-Testing Strategie & Umfassendes manuelles Testen über 50+ Test-Cases sicherte Qualität ohne Komplexität \\
\hline
Schlanke Architektur & Minimale Dependencies (nur Flask, pandas, Stripe/PayPal) ermöglichten schnelle Entwicklung und einfache Deployment \\
\hline
\end{tabular}
\end{table}

\section{Zielererreichung}
\label{sec:tab:ziele}

\begin{table}[H]
\centering
\caption{Erreichte Ziele vs. Aufgabenstellung}
\label{tab:ziele}
\begin{tabular}{|l|l|l|}
\hline
\textbf{Anforderung} & \textbf{Status} & \textbf{Implementierung} \\
\hline
Benutzerregistrierung und Login & $\checkmark$ & Vollständig mit PBKDF2 Hashing \\
\hline
Produktkatalog & $\checkmark$ & Komplett mit Bildern und Kategorien \\
\hline
Warenkorb-Verwaltung & $\checkmark$ & Session-basiert, AJAX-enabled \\
\hline
Zahlungsintegration & $\checkmark$ & Stripe und PayPal delegiert \\
\hline
Bestellverwaltung & $\checkmark$ & Vollständig mit Status-Tracking \\
\hline
Admin-Oberfläche & $\checkmark$ & Produkt- und Bestell-Management \\
\hline
DSGVO-Compliance & $\checkmark$ & Datenexport, Löschung, Audit-Logging \\
\hline
Sicherheit & $\checkmark$ & CSRF, XSS, SQLi Schutz \\
\hline
Testing & $\checkmark$ & 50+ E2E Test Cases, Unit Tests \\
\hline
\end{tabular}
\end{table}

\newpage
\chapter{Bilder}
\label{chap:bilder}

\section{UI-Mockups und Screenshots}
\label{sec:ui_screenshots}

\subsection{Startseite des Webshops}
\label{subsec:img_startseite}

\begin{figure}[H]
\centering
\framebox{
\begin{minipage}{0.9\textwidth}
\includegraphics[width=1\textwidth]{Images/mainPage.png}\\
\small Abbildung: Hauptseite des Onlineshops mit Produktgrid, Suchfeld und Filtermöglichkeiten
\end{minipage}
}
\caption{Startseite des Webshops}
\label{fig:startseite}
\end{figure}

\subsection{Login-Seite}
\label{subsec:img_login}

\begin{figure}[H]
\centering
\framebox{
\begin{minipage}{0.9\textwidth}
\includegraphics[width=1\textwidth]{Images/loginPage.png}\\
\small Abbildung: Login-Formular mit E-Mail und Passwort
\end{minipage}
}
\caption{Login-Seite}
\label{fig:login}
\end{figure}

\subsection{Produktdetail-Seite}
\label{subsec:img_produktdetail}

\begin{figure}[H]
\centering
\framebox{
\begin{minipage}{0.9\textwidth}
\includegraphics[width=1\textwidth]{Images/productDetailPage.png}\\
\small Abbildung: Produktdetail-Seite mit Bild, Beschreibung und Warenkorb-Button
\end{minipage}
}
\caption{Produktdetail-Seite}
\label{fig:produktdetail}
\end{figure}

\subsection{Warenkorb-Seite}
\label{subsec:img_warenkorb}

\begin{figure}[H]
\centering
\framebox{
\begin{minipage}{0.9\textwidth}
\includegraphics[width=1\textwidth]{Images/warenkorbPage.png}\\
\small Abbildung: Warenkorb mit Artikelübersicht und Checkout-Button
\end{minipage}
}
\caption{Warenkorb-Seite}
\label{fig:warenkorb}
\end{figure}

\subsection{Checkout-Seite}
\label{subsec:img_checkout}

\begin{figure}[H]
\centering
\framebox{
\begin{minipage}{0.9\textwidth}
\includegraphics[width=1\textwidth]{Images/checkoutPage.png}\\
\small Abbildung: Checkout-Formular mit Kundendaten und Zahlungsoptionen
\end{minipage}
}
\caption{Checkout-Seite}
\label{fig:checkout}
\end{figure}

\subsection{Bestellbestätigungsseite}
\label{subsec:img_bestaetigung}

\begin{figure}[H]
\centering
\framebox{
\begin{minipage}{0.9\textwidth}
\includegraphics[width=1\textwidth]{Images/orderConfirmPage.png}\\
\small Abbildung: Bestellbestätigungsseite mit Bestellnummer
\end{minipage}
}
\caption{Bestellbestätigungsseite}
\label{fig:bestaetigung}
\end{figure}

\subsection{Bestellungshistorie}
\label{subsec:img_bestellungen}

\begin{figure}[H]
\centering
\framebox{
\begin{minipage}{0.9\textwidth}
\includegraphics[width=1\textwidth]{Images/orderHistoryPage.png}\\
\small Abbildung: Dashboard mit Bestellungshistorie und Status
\end{minipage}
}
\caption{Bestellungshistorie}
\label{fig:bestellungen}
\end{figure}

\subsection{Admin-Produktverwaltung}
\label{subsec:img_admin_products}

\begin{figure}[H]
\centering
\framebox{
\begin{minipage}{0.9\textwidth}
\includegraphics[width=1\textwidth]{Images/productAdminPage.png}\\
\small Abbildung: Admin-Interface mit Produktverwaltung
\end{minipage}
}
\caption{Admin-Produktverwaltung}
\label{fig:admin_products}
\end{figure}

\section{Architektur-Diagramme}
\label{sec:architektur_diagramme}

\subsection{Schichtenmodell}
\label{subsec:img_schichten}

\begin{figure}[H]
\centering
\framebox{
\begin{minipage}{0.9\textwidth}
\includegraphics[width=1\textwidth]{Images/schichtenmodell.png}\\
\small Abbildung: 4-Schichten-Modell mit Presentation Layer (Flask Routes, Jinja2 Templates), Business Logic Layer (Services, Checkout), Data Access Layer (Hybrid Backend, Storage) und Data Storage Layer (SQLite, CSV)
\end{minipage}
}
\caption{Architektur-Schichtenmodell des Onlineshops}
\label{fig:schichten}
\end{figure}

\subsection{Entity-Relationship-Diagramm}
\label{subsec:img_er_diagramm}

\begin{figure}[H]
\centering
\framebox{
\begin{minipage}{0.9\textwidth}
\includegraphics[width=1\textwidth]{Images/er_diagramm.png}\\
\small Abbildung: Komplettes ERD mit 5 Entitäten (USER, PRODUCT, ORDER, USER\_CONSENT, AUDIT\_LOG) und ihren Beziehungen (Primärschlüssel, Fremdschlüssel, 1:n-Relationen)
\end{minipage}
}
\caption{Entity-Relationship-Diagramm des Onlineshops}
\label{fig:er_diagramm}
\end{figure}

\end{document}
