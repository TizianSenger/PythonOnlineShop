%!TEX encoding = UTF-8 Unicode
% Projektbericht LaTeX-Vorlage
% Zu bearbeitende Aufgabe: Konzeption und Umsetzung eines einfachen Onlineshops

\documentclass[12pt, a4paper, ngerman]{report}

% Pakete
\usepackage[utf8]{inputenc}
\usepackage[ngerman]{babel}
\usepackage[left=2cm, right=2cm, top=2cm, bottom=2cm]{geometry}
\usepackage{graphicx}
\usepackage{xcolor}
\usepackage{hyperref}
\usepackage{setspace}
\usepackage{fancyhdr}
\usepackage{float}
\usepackage{array}
\usepackage{longtable}
\usepackage{booktabs}
\usepackage{multirow}
\usepackage{makeidx}
\usepackage{tocloft}
\usepackage{caption}
\usepackage{subcaption}
\usepackage{verbatim}

% Zeilenabstand 1,5
\onehalfspacing

% Schriftarten
\usepackage{times}

% Seitenstil
\fancypagestyle{plain}{%
    \renewcommand{\headrulewidth}{0pt}%
    \renewcommand{\footrulewidth}{0pt}%
    \fancyhf{}%
    \fancyfoot[C]{\thepage}
}
\pagestyle{plain}

% Titel und Autor
\title{Konzeption und Umsetzung eines einfachen Onlineshops}
\author{Verfasser}
\date{\today}

\begin{document}

% ============================================================================
% TITELBLATT
% ============================================================================
\begin{titlepage}
    \begin{center}
        \vspace*{2cm}
        
        {\LARGE \textbf{Projektbericht}}
        
        \vspace{1.5cm}
        
        {\Large Konzeption und Umsetzung eines einfachen Onlineshops}
        
        \vspace{2cm}
        
        {\large Aufgabenstellung 2}
        
        \vspace{3cm}
        
        \vfill
        
        {\large \textbf{Verfasser:}} \\
        {\large [Name]} \\
        Matrikelnummer: [XXXXXXXXXXXX] \\
        
        \vspace{1cm}
        
        {\large \textbf{Kursbezeichnung:}} \\
        {\large [Kurs-Code]} \\
        
        \vspace{1cm}
        
        {\large \textbf{Studiengang:}} \\
        {\large [Studiengang]} \\
        
        \vspace{1cm}
        
        {\large \textbf{Tutor:}} \\
        {\large [Tutor Name]} \\
        
        \vspace{1cm}
        
        {\large \textbf{Abgabedatum:}} \\
        {\large \today}
        
        \vspace{2cm}
        
        IU Internationale Hochschule GmbH
        
    \end{center}
\end{titlepage}

% Seitennummerierung zurücksetzen
\setcounter{page}{1}
\pagenumbering{roman}

% ============================================================================
% INHALTSVERZEICHNIS
% ============================================================================
\tableofcontents
\newpage

% ============================================================================
% ABBILDUNGSVERZEICHNIS
% ============================================================================
\listoffigures
\newpage

% ============================================================================
% TABELLENVERZEICHNIS
% ============================================================================
\listoftables
\newpage

% ============================================================================
% ABKÜRZUNGSVERZEICHNIS
% ============================================================================
\chapter*{Abkürzungsverzeichnis}
\addcontentsline{toc}{chapter}{Abkürzungsverzeichnis}

\textit{(Tabelle 1 - siehe Anhang)}\newline

\newpage

% ============================================================================
% TEXTTEIL
% ============================================================================
\setcounter{page}{1}
\pagenumbering{arabic}

% ============================================================================
% KAPITEL 1: EINLEITUNG UND ANFORDERUNGSANALYSE
% ============================================================================
\chapter{Einleitung und Anforderungsanalyse}
\label{chap:einleitung}

\section{Problemstellung und Zielsetzung}
\label{sec:problemstellung}

Der digitale Handel nimmt global eine immer größere Rolle ein. Viele Unternehmen benötigen sichere, benutzerfreundliche und rechtskonforme E-Commerce-Lösungen, um ihre Produkte online zu verkaufen. Eine solche Lösung erfordert nicht nur technische Kompetenz, sondern auch ein tiefes Verständnis für Sicherheit, Datenschutz und regulatorische Anforderungen wie die Datenschutz-Grundverordnung (DSGVO).

Das Ziel dieses Projekts ist die Konzeption und Implementierung eines funktionsfähigen Onlineshops als Minimum Viable Product (MVP). Der entwickelte Shop soll folgende zentrale Anforderungen erfüllen:

\begin{itemize}
    \item Verwaltung eines Produktkatalogs mit Kategorisierung und Detailanzeigen
    \item Sichere Benutzer-Authentifizierung und Verwaltung
    \item Warenkorb- und Checkout-Funktionalität
    \item Bestellungsverwaltung und Orderverfolgung
    \item Administrator-Interface für die Produktpflege
    \item Datenschutz und Compliance gemäß DSGVO und internationalen Standards
    \item Robuste Datenspeicherung mit Fallback-Mechanismen
\end{itemize}

Der Shop wird mit modernen, skalierbaren Technologien umgesetzt und berücksichtigt von Anfang an Sicherheits- und Datenschutzaspekte.

\section{Zielgruppe und Rollen}
\label{sec:zielgruppe}

\subsection{Zielgruppen}
\label{subsec:zielgruppen}

Der entwickelte Onlineshop adressiert zwei primäre Zielgruppen:

\begin{enumerate}
    \item \textbf{Endkunden (B2C):} Diese Benutzergruppe durchstöbert den Produktkatalog, recherchiert Produktdetails, legt Artikel in den Warenkorb und schließt Bestellungen ab. Die Zielgruppe erwartet eine intuitive, sichere und schnelle Kaufabwicklung.
    
    \item \textbf{Shop-Administratoren:} Diese Benutzergruppe ist verantwortlich für die Verwaltung des Produktkatalogs, darunter das Hinzufügen, Bearbeiten und Löschen von Produkten. Administratoren können außerdem die Bestellungen überwachen und deren Status aktualisieren.
\end{enumerate}

\subsection{Benutzerrollen}
\label{subsec:rollen}

Das System implementiert zwei distinct Benutzerrollen mit unterschiedlichen Zugriffrechten (siehe \autoref{tab:benutzerrollen}).

\section{Use Cases und Anforderungen}
\label{sec:usecases}

\subsection{Funktionale Anforderungen}
\label{subsec:funktionale}

Die funktionalen Anforderungen definieren die konkreten Funktionalitäten, die der Shop bereitstellen muss:

\paragraph{Benutzerverwaltung}
\begin{itemize}
    \item Benutzer können sich mit E-Mail und Passwort registrieren
    \item Passwörter werden mit kryptografischen Hash-Funktionen (Werkzeug: \texttt{werkzeug.security}) gespeichert
    \item Benutzer können sich anmelden und abmelden
    \item Benutzerdaten können aktualisiert werden
    \item Admin-Registrierung ist durch eine PIN geschützt
\end{itemize}

\paragraph{Produktverwaltung}
\begin{itemize}
    \item Produkte können in Kategorien organisiert werden
    \item Der Katalog zeigt Produktlisten mit Name, Preis, Kategorie und Verfügbarkeit
    \item Detailseiten bieten erweiterte Informationen und Produktbilder
    \item Administratoren können Produkte erstellen, bearbeiten und löschen
    \item Produktbilder können hochgeladen und verwaltet werden
\end{itemize}

\paragraph{Warenkorb und Bestellung}
\begin{itemize}
    \item Benutzer können Produkte in den Warenkorb legen und die Menge anpassen
    \item Der Warenkorb wird in der Session gespeichert
    \item Die Checkout-Funktion sammelt Kundeninformationen
    \item Bestellungen werden persistent gespeichert
    \item Benutzer können ihre Bestellungshistorie einsehen
    \item Administratoren können den Status von Bestellungen ändern
\end{itemize}

\paragraph{DSGVO-Funktionalitäten}
\begin{itemize}
    \item Benutzer müssen der Datenschutzerklärung und den AGB zustimmen
    \item Einwilligungen zu Marketing- und Analytics-Cookies können granular gesteuert werden
    \item Benutzer können ihre persönlichen Daten exportieren (Art. 15 DSGVO)
    \item Benutzer können ihr Konto löschen (Art. 17 DSGVO, Recht auf Vergessenwerden)
    \item Audit-Logs dokumentieren alle kritischen Operationen
\end{itemize}

\subsection{Nicht-funktionale Anforderungen}
\label{subsec:nichtfunktional}

Nicht-funktionale Anforderungen beschreiben Qualitätsmerkmale und Betriebscharakteristiken:

\paragraph{Sicherheit}
\begin{itemize}
    \item Sichere Authentifizierung mit Passwort-Hashing
    \item Session-Management zur Verhinderung von Hijacking
    \item Schutz vor häufigen Angriffsmustern (XSS, CSRF, SQL-Injection)
    \item Keine Speicherung von Zahlungsdaten (Delegation an Zahlungsdienstleister)
    \item Sichere Datei-Uploads mit Validierung
\end{itemize}

\paragraph{Performance}
\begin{itemize}
    \item Schnelle Ladezeiten für Produktseiten (Ziel: < 2 Sekunden)
    \item Effiziente Datenbankabfragen
    \item Caching wo sinnvoll
\end{itemize}

\paragraph{Wartbarkeit und Skalierbarkeit}
\begin{itemize}
    \item Klare Modularität und Separation of Concerns
    \item Dokumentierter Code mit Kommentaren
    \item Möglichkeit zur Skalierung auf größere Datenmengen
    \item Hybrid-Backend für Fallback-Sicherheit
\end{itemize}

\paragraph{Datenschutz und Compliance}
\begin{itemize}
    \item Einhaltung der DSGVO-Anforderungen
    \item Transparente Datenpolitik und Impressum
    \item Audit-Logging für Compliance-Audits
    \item Sichere Zahlungsabwicklung über externe Provider (PCI-DSS Delegierung)
\end{itemize}

\section{Struktur des Projektberichts}
\label{sec:struktur}

Der vorliegende Projektbericht ist wie folgt aufgebaut:

\begin{description}
    \item[\textbf{Kapitel 1}] bietet eine Einführung in die Problemstellung, die Zielgruppen, Use Cases und Anforderungen des Projekts.
    
    \item[\textbf{Kapitel 2}] behandelt die Zahlungsabwicklung und die Einhaltung regulatorischer Anforderungen (DSGVO, PCI-DSS, PSD2).
    
    \item[\textbf{Kapitel 3}] präsentiert das UI-Design mit Mockups und das Datenmodell mit ER-Diagramm.
    
    \item[\textbf{Kapitel 4}] rechtfertigt die Technologieentscheidungen (Python, Flask, SQLite) und erörtert Alternativen.
    
    \item[\textbf{Kapitel 5}] erläutert die Softwarearchitektur, Designprinzipien und Sicherheitsmaßnahmen.
    
    \item[\textbf{Kapitel 6}] dokumentiert die Implementierung des MVP mit Fokus auf Projektplanung, Kernfunktionalitäten und aufgetretene Herausforderungen.
    
    \item[\textbf{Kapitel 7}] beschreibt Testing- und Qualitätssicherungsstrategien.
    
    \item[\textbf{Kapitel 8}] bietet eine kritische Reflexion des Projekts, identifiziert Lernpunkte und Verbesserungspotenziale.
    
    \item[\textbf{Kapitel 9}] zieht Fazit und präsentiert Ausblicke auf zukünftige Entwicklungen.
\end{description}

% ============================================================================
% KAPITEL 2: ZAHLUNGSABWICKLUNG UND COMPLIANCE
% ============================================================================
\chapter{Zahlungsabwicklung und Compliance}
\label{chap:zahlungsabwicklung}

\section{Zahlungsabwicklung}
\label{sec:zahlungssysteme}

\subsection{Gewählter Zahlungsanbieter und Begründung}
\label{subsec:zahlungsanbieter}

Der implementierte Onlineshop unterstützt zwei etablierte Zahlungsdienstleister:

\begin{enumerate}
    \item \textbf{Stripe} — Ein führendes Payment-Processing-System, das weltweit Millionen von Transaktionen abwickelt
    \item \textbf{PayPal} — Ein bewährter Zahlungsdienstleister mit starkem Vertrauen bei Kunden
\end{enumerate}

\paragraph{Begründung der Wahl:}

Die Auswahl dieser zwei Anbieter beruht auf mehreren Faktoren:

\begin{itemize}
    \item \textbf{Sicherheit:} Beide Anbieter sind PCI-DSS-zertifiziert (Level 1 bzw. 2) und delegieren die Verantwortung für Zahlungsdatenschutz
    \item \textbf{Verbreitung:} Sie genießen hohe Akzeptanz bei Kunden und ermöglichen Direktzahlungen sowie die Nutzung gespeicherter Zahlungsmethoden
    \item \textbf{Compliance:} Beide erfüllen die Payment Services Directive 2 (PSD2) und ermöglichen Strong Customer Authentication (SCA)
    \item \textbf{Entwicklerfreundlichkeit:} Umfangreiche APIs und SDKs für schnelle Integration
    \item \textbf{Testing-Umgebungen:} Beide bieten Sandbox-Umgebungen für Entwicklung und Tests ohne echte Zahlungen
\end{itemize}

Für die MVP-Implementierung wurden beide Systeme mit Test-Credentials konfiguriert, sodass die Integration erprobt werden kann, ohne echte Zahlungen zu verarbeiten.

\subsection{Integration und Sicherheit}
\label{subsec:zahlung_integration}

\paragraph{Architektur der Zahlungsintegration:}

Der Checkout-Prozess folgt diesem Ablauf:

\begin{enumerate}
    \item Der Benutzer füllt Checkout-Formular aus (Name, Adresse, Zahlungsmethode)
    \item Der Shop erstellt einen Bestelldatensatz mit Status \texttt{pending}
    \item Die Anfrage wird an Stripe oder PayPal weitergeleitet
    \item Der Benutzer wird auf die externe Zahlungsplattform umgeleitet
    \item Nach erfolgreicher Zahlungsbestätigung wird der Benutzer zurück zum Shop geleitet
    \item Der Bestellstatus wird auf \texttt{paid} aktualisiert
    \item Der Warenkorb wird geleert und eine Bestellbestätigung angezeigt
\end{enumerate}

\paragraph{Sicherheitsmaßnahmen:}

\begin{itemize}
    \item \textbf{Keine Speicherung von Zahlungsdaten:} Der Shop speichert keine Kreditkartendaten oder sensitive Zahlungsinformationen. Diese Daten werden direkt an Stripe/PayPal übertragen
    \item \textbf{HTTPS-Verschlüsselung:} Alle Zahlungsvorgänge erfolgen über verschlüsselte Verbindungen
    \item \textbf{Metadata-Tracking:} Zahlungs-IDs und Provider-IDs werden zur Audit-Spur gespeichert
    \item \textbf{Session-Management:} Nach erfolgreicher Zahlung wird die Session verwaltet, um unauthorized resubmission zu verhindern
\end{itemize}

\section{Compliance-Anforderungen}
\label{sec:compliance}

\subsection{DSGVO (Datenschutz-Grundverordnung)}
\label{subsec:dsgvo}

Die Datenschutz-Grundverordnung (DSGVO) ist die rechtliche Grundlage für den Datenschutz in der Europäischen Union und regelt die Verarbeitung personenbezogener Daten. Der Onlineshop implementiert folgende DSGVO-Anforderungen:

\paragraph{Rechtsgrundlagen (Art. 6 DSGVO):}
\begin{itemize}
    \item \textbf{Registrierung:} Explizite Einwilligung des Benutzers (Art. 6 Abs. 1 a DSGVO)
    \item \textbf{Bestellabwicklung:} Vertragserfüllung (Art. 6 Abs. 1 b DSGVO)
    \item \textbf{Zahlungsverarbeitung:} Rechtliche Verpflichtung (Art. 6 Abs. 1 c DSGVO)
\end{itemize}

\paragraph{Betroffenenrechte (Art. 12-22 DSGVO):}

Der Shop implementiert die folgenden Rechte (Details siehe \autoref{tab:dsgvo} im Anhang):

\paragraph{Datenschutzrichtlinien im Shop:}

\begin{itemize}
    \item \textbf{Privacy Policy:} Eine umfassende Datenschutzerklärung ist verfügbar und informiert über Datenerfassung, Verarbeitung und Rechte
    \item \textbf{Cookie-Banner:} Ein Consent-Banner informiert Benutzer über die Nutzung von Cookies und Tracking
    \item \textbf{Impressum:} Das gesetzliche Impressum ist gemäß TMG verfügbar
    \item \textbf{Bedingungen:} Allgemeine Geschäftsbedingungen (AGB) müssen vor der Registrierung akzeptiert werden
\end{itemize}

\paragraph{Datenminimierung (Art. 5 Abs. 1 c DSGVO):}

Der Shop erhebt nur die minimal notwendigen Daten:
\begin{itemize}
    \item Für Registrierung: Name, E-Mail, Passwort
    \item Für Bestellung: Lieferadresse, Rechnungsadresse
    \item Für Audit: Logging von kritischen Operationen (zeitgestempel, User-ID, Aktion)
\end{itemize}

\subsection{PCI-DSS und PSD2}
\label{subsec:pci_psd2}

\paragraph{Payment Card Industry Data Security Standard (PCI-DSS):}

PCI-DSS ist ein internationaler Standard zur Sicherung von Zahlungskartendaten. Der Shop erkennt an, dass die Verarbeitung von Zahlungsdaten eine hohe Verantwortung trägt.

\textbf{Compliance-Strategie:} Der Shop delegiert die Verantwortung vollständig an PCI-DSS-zertifizierte Anbieter (Stripe Level 1, PayPal Level 2). Der Shop selbst speichert oder verarbeitet \textbf{keine} Kreditkartendaten, Kontodetails oder andere sensitive Zahlungsinformationen. Dies reduziert das Compliance-Risiko erheblich.

\paragraph{Payment Services Directive 2 (PSD2):}

Die PSD2-Richtlinie regelt Zahlungsdienste in der EU und fordert unter anderem:

\begin{itemize}
    \item \textbf{Strong Customer Authentication (SCA):} Zwei-Faktor-Authentifizierung bei Zahlungen über €30 (wird durch Stripe/PayPal bereitgestellt)
    \item \textbf{Sichere Kommunikation:} Verschlüsselte Datenübertragung (HTTPS)
    \item \textbf{Transparenz:} Klare Informationen über Gebühren und Bedingungen
    \item \textbf{Haftung:} Regelung der Haftung bei unauthorized transactions
\end{itemize}

Der Shop implementiert diese Anforderungen durch die Nutzung von Stripe und PayPal, die als autorisierte Payment Service Provider nach PSD2 agieren.

\subsection{Umsetzung der Compliance-Anforderungen}
\label{subsec:umsetzung_compliance}

\paragraph{Audit-Logging und Transparenz:}

Ein zentrales Compliance-Instrument ist das Audit-Logging. Der Shop protokolliert alle kritischen Operationen:

\textit{Beispiel Audit-Log-Einträge:}
\begin{center}
\texttt{USER\_REGISTRATION, USER\_LOGIN, ORDER\_CREATED, DATA\_EXPORT, USER\_DELETED}
\end{center}

Jedes Audit-Log-Eintrag enthält:
\begin{itemize}
    \item Event Type (z.B. \texttt{USER\_REGISTRATION})
    \item Timestamp (ISO-Format)
    \item User ID und E-Mail
    \item Beschreibung der Aktion
    \item IP-Adresse des Benutzers
    \item Zusätzliche Details (z.B. Rolle bei Registrierung)
\end{itemize}

Diese Logs ermöglichen:
\begin{itemize}
    \item \textbf{Compliance-Audits:} Nachvollziehen, welche Operationen wann durchgeführt wurden
    \item \textbf{Sicherheitsuntersuchungen:} Identifizierung verdächtiger Aktivitäten
    \item \textbf{DSGVO-Anforderungen:} Nachweis der Datenverarbeitung für Betroffenenrechte
\end{itemize}

\paragraph{Datenexport und Löschung (DSGVO-Rechte):}

Der Shop implementiert zwei zentrale DSGVO-Funktionalitäten:

\begin{itemize}
    \item \textbf{Datenexport (Art. 15):} Benutzer können ihre Daten in JSON-Format herunterladen, einschließlich:
    \begin{itemize}
        \item Benutzerprofil (Name, E-Mail, Rolle)
        \item Alle Bestellungen und Bestellpositionen
        \item Alle Einwilligungen und Consent-Einträge
        \item Audit-Logs der eigenen Aktivitäten
    \end{itemize}
    
    \item \textbf{Kontolöschung (Art. 17):} Benutzer können die Löschung ihres Kontos anfordern, was zur:
    \begin{itemize}
        \item Löschung des Benutzerprofils führt
        \item Anonymisierung historischer Bestelldaten (um Compliance mit Aufbewahrungsfristen zu ermöglichen)
        \item Löschung aller Einwilligungen und Consent-Daten
    \end{itemize}
\end{itemize}

\paragraph{Encryption und Datenschutz at Rest:}

\begin{itemize}
    \item \textbf{Passwörter:} Alle Passwörter werden mit \texttt{werkzeug.security.generate\_password\_hash} gehasht (PBKDF2)
    \item \textbf{Datenbank:} SQLite-Daten können mit zusätzlicher Verschlüsselung geschützt werden (SQLCipher)
    \item \textbf{Sensible Konfigurationen:} API-Keys und Secrets werden über \texttt{.env}-Dateien verwaltet und nicht in den Code committed
\end{itemize}

% ============================================================================
% KAPITEL 3: UI-DESIGN UND DATENMODELL
% ============================================================================
\chapter{UI-Design und Datenmodell}
\label{chap:ui_datenmodell}

\section{UI-Design und Mockups}
\label{sec:ui_design}

Die Benutzeroberfläche des Onlineshops wurde mit dem Ziel einer intuitiven, benutzerfreundlichen Gestaltung entwickelt. Das Design folgt modernen Web-Standards und ist responsive, d.h., es passt sich automatisch an verschiedene Bildschirmgrößen (Desktop, Tablet, Mobile) an.

\subsection{Wireframes und Mockups}
\label{subsec:wireframes}

Der Shop wurde mit HTML, CSS und JavaScript implementiert. Die Template-Engine Jinja2 ermöglicht dynamische Inhalte. Im Folgenden sind die Hauptseiten des Shops beschrieben:

\paragraph{Startseite (Index):}

Die Startseite ist das Herzstück des Shops und präsentiert den Produktkatalog mit folgenden Funktionen:

\begin{itemize}
    \item \textbf{Produktgrid:} Alle verfügbaren Produkte werden in einer responsiven Grid-Ansicht angezeigt
    \item \textbf{Produktkarte:} Jede Karte zeigt Produktbild, Name, Kategorie, Preis und einen \textit{In den Warenkorb}-Button
    \item \textbf{Suchfunktion:} Benutzer können nach Produktnamen suchen
    \item \textbf{Filterung:} Nach Kategorie, Mindest- und Maximalpreis filterbar
    \item \textbf{Warenkorb-Counter:} Zeigt aktuelle Anzahl der Artikel im Warenkorb
\end{itemize}

\begin{figure}[H]
\centering
\framebox{
\begin{minipage}{0.9\textwidth}
\textit{[Screenshot der Startseite einfügen]}\\
\small Abbildung: Hauptseite des Onlineshops mit Produktgrid, Suchfeld und Filtermöglichkeiten
\end{minipage}
}
\caption{Startseite des Webshops}
\label{fig:startseite}
\end{figure}

\paragraph{Login und Registrierung:}

Benutzer müssen sich registrieren oder anmelden, um zu bestellen:

\begin{itemize}
    \item \textbf{Registrierungsformular:} E-Mail, Name, Passwort, Rolle (User/Admin), optional Admin-PIN
    \item \textbf{Einwilligungen:} Checkboxen für Datenschutz, AGB, Marketing, Analytics
    \item \textbf{Loginformular:} E-Mail und Passwort
    \item \textbf{Session-Management:} Nach erfolgreicher Anmeldung wird eine Session erstellt
\end{itemize}

\begin{figure}[H]
\centering
\framebox{
\begin{minipage}{0.9\textwidth}
\textit{[Screenshot der Login-Seite einfügen]}\\
\small Abbildung: Login-Formular mit E-Mail und Passwort
\end{minipage}
}
\caption{Login-Seite}
\label{fig:login}
\end{figure}

\paragraph{Produktdetail-Seite:}

Wenn ein Benutzer auf ein Produkt klickt, wird die Detailseite angezeigt:

\begin{itemize}
    \item \textbf{Produktbild:} Hochauflösendes Bild des Produkts
    \item \textbf{Produktinformationen:} Name, Preis, Kategorie, Lagerstatus
    \item \textbf{Beschreibung:} Detaillierte Produktbeschreibung
    \item \textbf{Menge-Selector:} Benutzer kann Anzahl auswählen
    \item \textbf{In den Warenkorb Button:} AJAX-basierter Button (keine Seite wird neu geladen)
\end{itemize}

\begin{figure}[H]
\centering
\framebox{
\begin{minipage}{0.9\textwidth}
\textit{[Screenshot der Produktdetail-Seite einfügen]}\\
\small Abbildung: Produktdetail-Seite mit Bild, Beschreibung und Warenkorb-Button
\end{minipage}
}
\caption{Produktdetail-Seite}
\label{fig:produktdetail}
\end{figure}

\paragraph{Warenkorb:}

Der Warenkorb-View zeigt alle hinzugefügten Artikel:

\begin{itemize}
    \item \textbf{Artikel-Liste:} Tabellarische Ansicht mit Produktname, Preis, Menge, Gesamtbetrag
    \item \textbf{Menge-Anpassung:} Benutzer kann Menge ändern oder Artikel entfernen
    \item \textbf{Gesamtpreis:} Summe aller Artikel
    \item \textbf{Checkout-Button:} Leads zu Checkout-Seite
\end{itemize}

\begin{figure}[H]
\centering
\framebox{
\begin{minipage}{0.9\textwidth}
\textit{[Screenshot der Warenkorb-Seite einfügen]}\\
\small Abbildung: Warenkorb mit Artikelübersicht und Checkout-Button
\end{minipage}
}
\caption{Warenkorb-Seite}
\label{fig:warenkorb}
\end{figure}

\paragraph{Checkout:}

Das Checkout-Formular ist das Herzstück der Bestellabwicklung:

\begin{itemize}
    \item \textbf{Bestelldetails:} Zusammenfassung der Artikel und Gesamtpreis
    \item \textbf{Kundendaten:} Name, E-Mail, Lieferadresse, Rechnungsadresse
    \item \textbf{Zahlungsmethode:} Radio-Buttons zur Wahl zwischen Stripe und PayPal
    \item \textbf{Zahlungs-Integration:} Bei Klick auf \textit{Bezahlen} wird zur externen Zahlungsplattform weitergeleitet
\end{itemize}

\begin{figure}[H]
\centering
\framebox{
\begin{minipage}{0.9\textwidth}
\textit{[Screenshot der Checkout-Seite einfügen]}\\
\small Abbildung: Checkout-Formular mit Kundendaten und Zahlungsoptionen
\end{minipage}
}
\caption{Checkout-Seite}
\label{fig:checkout}
\end{figure}

\paragraph{Bestellbestätigung:}

Nach erfolgreicher Zahlung wird eine Bestätigungsseite angezeigt:

\begin{itemize}
    \item \textbf{Erfolgs-Nachricht:} Danksagung und Bestätigung der Bestellung
    \item \textbf{Bestellnummer:} Eindeutige ID für Verfolgung
    \item \textbf{Bestelldetails:} Zusammenfassung der Artikel und Adresse
    \item \textbf{Nächste Schritte:} Link zu Bestellungshistorie
\end{itemize}

\begin{figure}[H]
\centering
\framebox{
\begin{minipage}{0.9\textwidth}
\textit{[Screenshot der Bestätigungsseite einfügen]}\\
\small Abbildung: Bestellbestätigungsseite mit Bestellnummer
\end{minipage}
}
\caption{Bestellbestätigungsseite}
\label{fig:bestaetigung}
\end{figure}

\paragraph{Dashboard und Bestellungshistorie:}

Angemeldete Benutzer können ihre Bestellungshistorie einsehen:

\begin{itemize}
    \item \textbf{Bestellliste:} Tabelle mit allen bisherigen Bestellungen
    \item \textbf{Status-Anzeige:} Aktueller Status jeder Bestellung (bezahlt, in Bearbeitung, versendet, zugestellt)
    \item \textbf{Bestelldetails:} Klick auf eine Bestellung zeigt Artikel und Gesamtbetrag
    \item \textbf{Admin-Features:} Administratoren können Bestellstatus ändern
\end{itemize}

\begin{figure}[H]
\centering
\framebox{
\begin{minipage}{0.9\textwidth}
\textit{[Screenshot der Bestellungshistorie einfügen]}\\
\small Abbildung: Dashboard mit Bestellungshistorie und Status
\end{minipage}
}
\caption{Bestellungshistorie}
\label{fig:bestellungen}
\end{figure}

\paragraph{Admin-Interface:}

Administratoren haben Zugriff auf ein spezielles Interface:

\begin{itemize}
    \item \textbf{Produktverwaltung:} Tabelle mit allen Produkten
    \item \textbf{Produkt hinzufügen:} Formular für neues Produkt
    \item \textbf{Produkt bearbeiten:} Inline-Bearbeitung von Name, Preis, Kategorie, Beschreibung, Bildern
    \item \textbf{Produkt löschen:} Bestätigtes Löschen von Produkten
    \item \textbf{Bestellungsverwaltung:} Übersicht aller Bestellungen mit Status-Updates
\end{itemize}

\begin{figure}[H]
\centering
\framebox{
\begin{minipage}{0.9\textwidth}
\textit{[Screenshot der Admin-Produktverwaltung einfügen]}\\
\small Abbildung: Admin-Interface mit Produktverwaltung
\end{minipage}
}
\caption{Admin-Produktverwaltung}
\label{fig:admin_products}
\end{figure}

\subsection{Navigationskonzept}
\label{subsec:navigation}

Die Navigation des Shops folgt einem klaren, hierarchischen Struktur:

\begin{itemize}
    \item \textbf{Hauptnavigation (Header):} 
    \begin{itemize}
        \item Logo / Shop-Name (Link zur Startseite)
        \item Warenkorb-Icon mit Counter
        \item Login / Logout Button
        \item Dashboard (für angemeldete Benutzer)
        \item Admin-Panel (nur für Administratoren)
    \end{itemize}
    
    \item \textbf{Sekundäre Navigation (Footer):}
    \begin{itemize}
        \item Datenschutzerklärung
        \item Impressum
        \item AGB
        \item DSGVO-Rechte / Datenexport
        \item Kontakt
    \end{itemize}
    
    \item \textbf{Breadcrumbs:} Zeigen den aktuellen Ort im Shop (z.B. \texttt{Startseite > Kategorie > Produkt})
\end{itemize}

Die Navigation ist vollständig responsive und nutzt Mobile-Menüs auf kleinen Bildschirmen.

\section{Datenmodell}
\label{sec:datenmodell}

\subsection{Entitäten und Beziehungen}
\label{subsec:entitaeten}

Das Datenmodell des Onlineshops ist relational und besteht aus fünf Hauptentitäten:

\begin{enumerate}
    \item \textbf{User} — Speichert Benutzerdaten (Kunden und Administratoren)
    \item \textbf{Product} — Speichert Produktinformationen (Name, Preis, Kategorie, Beschreibung)
    \item \textbf{Order} — Speichert Bestellungen mit Gesamtbetrag und Status
    \item \textbf{OrderItem} — Speichert Bestellpositionen (welche Produkte, Menge, Preis pro Stück)
    \item \textbf{Consent \& AuditLog} — Speichern DSGVO-Compliance-Informationen
\end{enumerate}

\paragraph{Beziehungen:}

\begin{itemize}
    \item \textbf{User → Order:} Ein Benutzer kann viele Bestellungen haben (1:n)
    \item \textbf{Order → OrderItem:} Eine Bestellung enthält viele Bestellpositionen (1:n)
    \item \textbf{Product → OrderItem:} Ein Produkt kann in vielen Bestellpositionen vorkommen (1:n)
    \item \textbf{User → Consent:} Ein Benutzer hat viele Einwilligungen (1:n, z.B. Privacy, Marketing, Analytics)
    \item \textbf{User → AuditLog:} Ein Benutzer wird in vielen Audit-Logs dokumentiert (1:n)
\end{itemize}

\subsection{ER-Diagramm}
\label{subsec:er_diagramm}

Das folgende ER-Diagramm visualisiert das Datenbankschema:

\begin{figure}[H]
\centering
\framebox{
\begin{minipage}{0.9\textwidth}
\textit{[ER-Diagramm einfügen]}\\
\small Abbildung: Entity-Relationship-Diagramm mit Entitäten, Attributen und Beziehungen

Die Abbildung sollte folgende Elemente zeigen:
\begin{itemize}
    \item USER-Tabelle (id, name, email, password, role)
    \item PRODUCT-Tabelle (id, name, price, category, description, stock, images)
    \item ORDER-Tabelle (id, user\_id, total, status, created\_at, payment\_provider)
    \item ORDER\_ITEM-Tabelle (id, order\_id, product\_id, quantity, price)
    \item CONSENT-Tabelle (id, user\_id, consent\_type, value, created\_at)
    \item AUDIT\_LOG-Tabelle (id, user\_id, event\_type, action, ip\_address, created\_at)
\end{itemize}
\end{minipage}
}
\caption{Entity-Relationship-Diagramm des Onlineshops}
\label{fig:er_diagramm}
\end{figure}

\subsection{Datenbeschreibung}
\label{subsec:datenbeschreibung}

Folgende Tabellen werden im System verwendet:

\paragraph{USER-Tabelle:}

Speichert Benutzerinformationen für Authentifizierung und Profilverwaltung.

\textit{(Tabelle 2 - siehe Anhang)}\newline

\paragraph{PRODUCT-Tabelle:}

Speichert Produktinformationen für den Katalog.

\textit{(Tabelle 3 - siehe Anhang)}\newline

\paragraph{ORDER-Tabelle:}

Speichert Bestellungen mit Metadaten.

\textit{(Tabelle 4 - siehe Anhang)}\newline

\paragraph{CONSENT-Tabelle:}

Speichert Einwilligungen des Benutzers (DSGVO).

\textit{(Tabelle 5 - siehe Anhang)}\newline

\paragraph{AUDIT\_LOG-Tabelle:}

Speichert kritische Operationen für Compliance und Sicherheit.

\textit{(Tabelle 6 - siehe Anhang)}\newline

% ============================================================================
% KAPITEL 4: TECHNOLOGIEENTSCHEIDUNGEN
% ============================================================================
\chapter{Technologieentscheidungen}
\label{chap:technologie}

\section{Programmiersprache und Framework}
\label{sec:programmiersprache}

\subsection{Wahl: Python mit Flask}
\label{subsec:python_flask}

Für die Implementierung des Onlineshops wurde die Programmiersprache \textbf{Python} in Kombination mit dem Web-Framework \textbf{Flask} gewählt. Diese Wahl weicht von den in der Aufgabenstellung genannten Beispielen (PHP, Java) ab, ist jedoch gemäß Aufgabenstellung zulässig.

\paragraph{Python:}

Python ist eine moderne, interpretierte Programmiersprache mit folgenden Vorteilen:

\begin{itemize}
    \item \textbf{Lesbarkeit und Wartbarkeit:} Python-Code ist ausgesprochen lesbar und selbstdokumentierend. Die klare Syntax reduziert cognitive load beim Verstehen und Erweitern des Codes
    \item \textbf{Schnelle Entwicklung:} Die hohe Produktivität ermöglicht schnelle Prototypenentwicklung und iteratives Design
    \item \textbf{Reiches Ökosystem:} Tausende von Libraries und Packages (PyPI) vereinfachen häufige Aufgaben
    \item \textbf{Starke Community:} Umfangreiche Dokumentation, Tutorials und Forum-Support
    \item \textbf{Datenbankunterstützung:} Python hat exzellente ORM-Frameworks und native SQLite-Unterstützung
    \item \textbf{Testing:} Frameworks wie \texttt{pytest} ermöglichen einfaches, umfassendes Testing
\end{itemize}

\paragraph{Flask:}

Flask ist ein leichtgewichtiges, modulares Web-Framework mit dem Prinzip \textit{„Micro-framework"}:

\begin{itemize}
    \item \textbf{Minimalistisch:} Flask bietet die essentiellen Komponenten (Routing, Request/Response Handling, Session Management) ohne unnötige Komplexität
    \item \textbf{Flexibel:} Entwickler können beliebige Packages integrieren, ohne durch Framework-Vorgaben eingeengt zu sein
    \item \textbf{Template-Engine:} Jinja2 ermöglicht sichere, dynamische HTML-Generierung
    \item \textbf{Skalierbar:} Flask-Anwendungen lassen sich von MVP bis zu Production-Grade scaling entwickeln
    \item \textbf{Security:} Eingebaute Schutzmaßnahmen gegen CSRF, XSS und andere Angriffsmuster
\end{itemize}

Die Kombination Python + Flask ermöglicht eine schnelle, sichere und wartbare Implementierung des Onlineshops.

\subsection{Begründung und Alternativen}
\label{subsec:alternativen_framework}

\paragraph{Alternativen und ihre Bewertung:}

\textit{(Tabelle 7 - siehe Anhang)}\newline

Python + Flask wurde als \textbf{optimale Balance zwischen Entwicklungsgeschwindigkeit und Funktionalität} für das MVP gewählt.

\section{Frontend-Technologien}
\label{sec:frontend}

\subsection{HTML, CSS und JavaScript}
\label{subsec:html_css_js}

Das Frontend des Shops nutzt Standard-Webtechnologien ohne schwere Frameworks:

\paragraph{HTML5:}
\begin{itemize}
    \item Semantische HTML-Struktur mit \texttt{<header>}, \texttt{<main>}, \texttt{<footer>}
    \item Formulare für Login, Registrierung, Checkout
    \item Responsive Bilder mit \texttt{<picture>} und \texttt{srcset}
    \item Accessibility-Attribute (\texttt{aria-*}, \texttt{role})
\end{itemize}

\paragraph{CSS3:}
\begin{itemize}
    \item \textbf{Responsive Design:} Mobile-First Ansatz mit Media Queries
    \item \textbf{Flexbox \& CSS Grid:} Modernes Layout-System für flexible Designs
    \item \textbf{CSS Variables:} Zentrale Verwaltung von Farben, Schriftarten, Abständen
    \item \textbf{Transitions \& Animations:} Sanfte UX-Übergänge
\end{itemize}

\paragraph{JavaScript (Vanilla):}
\begin{itemize}
    \item \textbf{AJAX-Requests:} Warenkorb-Updates ohne Seite neu zu laden
    \item \textbf{Form Validation:} Client-seitige Validierung vor Submission
    \item \textbf{Event Handling:} Interaktive Elemente (Buttons, Filter, etc.)
    \item \textbf{Keine Abhängigkeiten:} Vanilla JS reduziert Komplexität, keine zusätzlichen Packages nötig
\end{itemize}

Diese Wahl ermöglicht schnelle Ladezeiten und minimale Abhängigkeiten.

\section{Datenspeicherung}
\label{sec:datenspeicherung}

\subsection{SQLite als primäres Speichersystem}
\label{subsec:sqlite}

\textbf{SQLite} wurde als primäres Speichersystem gewählt. SQLite ist eine serverlose, dateibasierte SQL-Datenbank mit folgenden Eigenschaften:

\paragraph{Vorteile von SQLite:}

\begin{itemize}
    \item \textbf{Einfache Einrichtung:} Keine separate Datenbank-Server-Installation erforderlich, nur eine Datei (\texttt{webshop.db})
    \item \textbf{ACID-Garantien:} Transaktionale Konsistenz und Zuverlässigkeit
    \item \textbf{SQL-Standard:} Vollständige SQL-Unterstützung, Standard-Abfragen funktionieren direkt
    \item \textbf{Embedded:} Die Datenbank läuft im gleichen Prozess wie die Anwendung
    \item \textbf{Perfekt für MVP:} Ideal für Prototypen und kleine bis mittlere Datenmengen
    \item \textbf{Python-Integration:} Nativer \texttt{sqlite3}-Support in der Python-Standardbibliothek
\end{itemize}

\paragraph{Limitierungen und Skalierungsgrenzen:}

SQLite hat Limitierungen bei sehr hohem Durchsatz:

\begin{itemize}
    \item \textbf{Schreib-Concurrency:} Nur eine Schreiboperation gleichzeitig (Lock-Mechanismus)
    \item \textbf{Dateigröße:} Theoretisch bis zu 281 TB, praktisch aber begrenzt durch Filesystem
    \item \textbf{Benutzer} (bis ca. 1000 gleichzeitig); für große Shops empfohlen: PostgreSQL oder MySQL
\end{itemize}

Für das MVP mit sinthetischen Testdaten ist SQLite vollkommen ausreichend.

\subsection{CSV als Fallback und Migration}
\label{subsec:csv}

Der Shop implementiert \textbf{CSV-Dateien} als Fallback-Speichermechanismus mit folgenden Dateien:

\begin{itemize}
    \item \texttt{data/csv/users.csv} — Benutzerdaten
    \item \texttt{data/csv/products.csv} — Produktinformationen
    \item \texttt{data/csv/orders.csv} — Bestelldaten
    \item \texttt{data/csv/user\_consents.csv} — DSGVO-Einwilligungen
    \item \texttt{data/logs/audit\_log.csv} — Audit-Logs
\end{itemize}

\paragraph{Zweck des CSV-Backends:}

\begin{itemize}
    \item \textbf{Menschenlesbar:} CSV-Dateien können mit jedem Text-Editor oder Tabellenkalkulation gelesen werden
    \item \textbf{Backup:} Redundante Datenspeicherung neben SQLite
    \item \textbf{Migration:} Automatisches Migrationsscript (\texttt{migrate\_csv\_to\_sqlite.py}) überführt CSV-Daten in SQLite
    \item \textbf{Testing:} CSV ermöglicht einfaches Seeding von Testdaten
    \item \textbf{Fallback:} Falls SQLite ausfällt, wird automatisch auf CSV zurückgegriffen (siehe Hybrid Backend)
\end{itemize}

\paragraph{Struktur beispielhafter CSV (users.csv):}

\begin{center}
\small
\texttt{id, name, email, password, role, created\_at} \\
\texttt{1, Max Mustermann, max@example.com, pbkdf2:..., user, 2026-01-01T10:00:00} \\
\texttt{2, Admin User, admin@example.com, pbkdf2:..., admin, 2026-01-01T10:00:00}
\end{center}

\subsection{Hybrid-Backend-Architektur}
\label{subsec:hybrid_backend}

Der Onlineshop implementiert ein innovatives \textbf{Hybrid-Backend}, das SQLite und CSV intelligent kombiniert:

\paragraph{Architektur-Übersicht:}

Das Hybrid Backend folgt diesem Datenfluss:

\begin{figure}[H]
\centering
\framebox{
\begin{minipage}{0.85\textwidth}
\textit{[Architektur-Diagramm: Hybrid Backend einfügen]}

\small Abbildung zeigt: Flask App → Hybrid Backend → SQLite (Primary) mit Fallback zu CSV (Backup)
\end{minipage}
}
\caption{Hybrid-Backend Datenfluss}
\label{fig:hybrid_backend}
\end{figure}

\paragraph{Funktionsweise:}

\begin{enumerate}
    \item \textbf{Schreib-Operationen:}
    \begin{itemize}
        \item Versuche zu SQLite zu schreiben
        \item Falls erfolgreich: Schreibe auch zu CSV als Backup
        \item Falls SQLite-Fehler: Schreibe nur zu CSV, protokolliere Fallback-Event
    \end{itemize}
    
    \item \textbf{Lese-Operationen:}
    \begin{itemize}
        \item Versuche von SQLite zu lesen
        \item Falls erfolgreich: Rückgabe der Daten
        \item Falls SQLite-Fehler: Fallback zu CSV, protokolliere Fallback-Event
    \end{itemize}
\end{enumerate}

\paragraph{Vorteile:}

\begin{itemize}
    \item \textbf{Redundanz:} Datenausfallsicherheit durch Backup
    \item \textbf{Robustheit:} Shop läuft weiter, auch wenn SQLite Probleme hat
    \item \textbf{Monitoring:} Fallback-Events werden protokolliert zur Überwachung
    \item \textbf{Debugging:} Einfaches Debugging durch einsehbare CSV-Dateien
\end{itemize}

Implementierung in \texttt{src/storage/hybrid\_backend.py} mit Methoden:
\begin{itemize}
    \item \texttt{\_try\_sqlite()} — Versucht SQLite-Operation
    \item \texttt{\_fallback\_to\_csv()} — Fällt zu CSV zurück bei Fehlern
    \item \texttt{get\_all\_products()}, \texttt{save\_user()}, etc. — Unified API
\end{itemize}

\section{Entwicklungs- und Betriebsumgebung}
\label{sec:umgebung}

\subsection{Lokale Entwicklungsumgebung}
\label{subsec:dev_umgebung}

Die Entwicklung erfolgt in einer lokalen Python-Umgebung mit folgender Konfiguration:

\paragraph{System-Anforderungen:}

\begin{itemize}
    \item \textbf{Betriebssystem:} Windows, macOS oder Linux
    \item \textbf{Python:} Version 3.8 oder höher
    \item \textbf{Speicher:} Mindestens 512 MB verfügbar
    \item \textbf{Festplatte:} Mindestens 500 MB für Abhängigkeiten
\end{itemize}

\paragraph{Entwicklungs-Setup:}

\begin{verbatim}
# 1. Virtual Environment erstellen
python -m venv venv

# 2. Aktivieren
source venv/bin/activate  # Linux/macOS
venv\Scripts\activate     # Windows

# 3. Dependencies installieren
pip install -r requirements.txt

# 4. Datenbank initialisieren
python src/storage/init_database.py

# 5. App starten
python src/app.py
\end{verbatim}

\paragraph{Flask Development Server:}

Die App läuft im Development Mode mit folgenden Features:

\begin{itemize}
    \item \textbf{Auto-Reload:} Code-Änderungen werden sofort neu geladen
    \item \textbf{Debugger:} Interaktiver Debugger bei Fehlern
    \item \textbf{Request Logging:} Alle HTTP-Requests werden protokolliert
    \item \textbf{Debug-Toolbar:} Optionale Flask-Debug-Toolbar für Profiling
\end{itemize}

\subsection{Dependencies und Packages}
\label{subsec:dependencies}

Das Projekt nutzt folgende externe Python-Packages (siehe \texttt{requirements.txt}):

\textit{(Tabelle 8 - siehe Anhang)}\newline

\paragraph{Optional für Erweiterungen:}

\begin{itemize}
    \item \textbf{SQLAlchemy:} ORM für komplexere Datenbank-Operationen
    \item \textbf{Celery + Redis:} Für asynchrone Task-Verarbeitung
    \item \textbf{gunicorn:} WSGI-Server für Production-Deployment
    \item \textbf{nginx:} Reverse-Proxy für Production
\end{itemize}

Diese Packages sind alle Open-Source und gut dokumentiert, ohne lizenzrechtliche Einschränkungen.

% ============================================================================
% KAPITEL 5: ARCHITEKTUR UND SOFTWARE-DESIGN
% ============================================================================
\chapter{Architektur und Software-Design}
\label{chap:architektur}

\section{Softwarearchitektur}
\label{sec:softwarearchitektur}

\subsection{Architekturübersicht}
\label{subsec:arch_overview}

Der Onlineshop folgt einer klaren, modularen Architektur mit Separation of Concerns. Die Architektur besteht aus vier Schichten:

\begin{figure}[H]
\centering
\framebox{
\begin{minipage}{0.9\textwidth}
\textit{[Architektur-Schichtenmodell einfügen]}

\small Abbildung zeigt die vier Schichten (von oben nach unten):
\begin{itemize}
    \item Presentation Layer (Flask Routes, Jinja2 Templates)
    \item Business Logic Layer (Services, Checkout)
    \item Data Access Layer (Hybrid Backend, Storage)
    \item Data Storage Layer (SQLite, CSV)
\end{itemize}
\end{minipage}
}
\caption{Architektur-Schichtenmodell des Onlineshops}
\label{fig:schichten}
\end{figure}

\subsection{Schichtenmodell}
\label{subsec:schichten}

\paragraph{Layer 1: Presentation Layer (Flask Routes)}

Die Präsentationsschicht wird durch Flask-Routes und Jinja2-Templates implementiert:

\begin{itemize}
    \item \textbf{Routes:} Alle HTTP-Endpoints sind in \texttt{src/app.py} und \texttt{src/api/checkout\_routes.py} definiert
    \item \textbf{Request Handling:} Flask verarbeitet GET/POST-Requests, validiert Input und leitet an Business-Logic weiter
    \item \textbf{Response Rendering:} Jinja2-Templates generieren dynamisches HTML
    \item \textbf{Session Management:} Flask-Sessions speichern Benutzer-ID, Warenkorb, etc.
    \item \textbf{Error Handling:} 404, 500 Fehler werden elegant behandelt
\end{itemize}

Hauptroutes:
\begin{itemize}
    \item \texttt{GET /} — Startseite mit Produktgrid
    \item \texttt{GET/POST /register} — Benutzer-Registrierung
    \item \texttt{GET/POST /login} — Authentifizierung
    \item \texttt{POST /add-to-cart} — AJAX Warenkorb-Update
    \item \texttt{GET /checkout} — Checkout-Formular
    \item \texttt{POST /checkout/create} — Bestellung erstellen
    \item \texttt{GET /admin/products} — Admin-Interface
\end{itemize}

\paragraph{Layer 2: Business Logic Layer (Services)}

Die Business Logic wird in Service-Modulen implementiert:

\begin{itemize}
    \item \textbf{Checkout Service:} \texttt{src/services/checkout.py}
    \begin{itemize}
        \item \texttt{create\_stripe\_session()} — Stripe Session erzeugen
        \item \texttt{create\_paypal\_order()} — PayPal Order erzeugen
        \item \texttt{save\_order()} — Bestellung in Backend speichern
    \end{itemize}
    
    \item \textbf{Authentication Logic:} In Routes implementiert
    \begin{itemize}
        \item Password Hashing mit \texttt{werkzeug.security.generate\_password\_hash}
        \item Password Verification mit \texttt{check\_password\_hash}
        \item Session Token Management
    \end{itemize}
    
    \item \textbf{Logging Service:} \texttt{src/utils/logging\_service.py}
    \begin{itemize}
        \item DSGVO-konformes Audit-Logging
        \item Event Types: USER\_REGISTRATION, ORDER\_CREATED, DATA\_EXPORT, etc.
    \end{itemize}
\end{itemize}

\paragraph{Layer 3: Data Access Layer (Hybrid Backend)}

Die Datenzugriff-Schicht wird durch das Hybrid Backend implementiert:

\begin{itemize}
    \item \textbf{HybridBackend:} \texttt{src/storage/hybrid\_backend.py}
    \begin{itemize}
        \item Unified Interface für Datenzugriff
        \item Versucht SQLite → Fallback zu CSV
        \item Methoden: \texttt{get\_all\_products()}, \texttt{save\_user()}, \texttt{create\_order()}, etc.
    \end{itemize}
    
    \item \textbf{SQLiteBackend:} \texttt{src/storage/sqlite\_backend.py}
    \begin{itemize}
        \item Direkter SQLite-Zugriff mit \texttt{sqlite3}
        \item CRUD-Operationen für alle Entitäten
    \end{itemize}
    
    \item \textbf{CSVBackend:} \texttt{src/storage/csv\_backend.py}
    \begin{itemize}
        \item Fallback-Speicher via CSV-Dateien
        \item Verwendet pandas für effiziente Dateioperationen
    \end{itemize}
\end{itemize}

\paragraph{Layer 4: Data Storage Layer}

Die physische Datenspeicherung erfolgt auf zwei Wegen:

\begin{itemize}
    \item \textbf{SQLite:} \texttt{data/webshop.db}
    \item \textbf{CSV-Dateien:} \texttt{data/csv/} mit users.csv, products.csv, orders.csv, user\_consents.csv
\end{itemize}

\subsection{Datenfluss}
\label{subsec:datenfluss}

\paragraph{Beispiel: Neuer Benutzer wird registriert}

\begin{enumerate}
    \item Benutzer füllt Registrierungs-Formular aus und klickt \textit{Registrieren}
    \item POST-Request geht an Flask Route \texttt{/register}
    \item Flask validiert Input (Name, Email, Passwort sind nicht leer)
    \item Flask validiert Einwilligungen (Privacy Policy, AGB müssen akzeptiert sein)
    \item Passwort wird gehasht: \texttt{password\_hash = generate\_password\_hash(password)}
    \item Benutzer-Objekt wird erstellt: \texttt{\{name, email, password\_hash, role='user'\}}
    \item Backend-Aufruf: \texttt{backend.save\_user(user)}
    \item HybridBackend versucht SQLite zu schreiben:
    \begin{itemize}
        \item SQLite INSERT erfolgreich
        \item User-ID wird zurückgegeben (z.B. UUID)
    \end{itemize}
    \item HybridBackend schreibt auch zu CSV als Backup
    \item Audit-Log wird erstellt: \texttt{AuditLogType.USER\_REGISTRATION}
    \item Einwilligungen werden gespeichert: \texttt{backend.save\_consent(user\_id, 'privacy\_policy', 'True')}
    \item Session wird gesetzt: \texttt{session["user"] = \{id, name, email, role\}}
    \item Benutzer wird zu Dashboard weitergeleitet
\end{enumerate}

\paragraph{Beispiel: Benutzer legt Produkt in Warenkorb}

\begin{enumerate}
    \item Benutzer klickt \textit{In den Warenkorb} Button
    \item JavaScript sendet AJAX-POST zu \texttt{/add-to-cart} mit \texttt{\{product\_id: 123, quantity: 1\}}
    \item Flask validiert dass Benutzer angemeldet ist (falls nicht: 401 Unauthorized)
    \item Warenkorb wird aus Session geholt: \texttt{cart = session.get("cart", [])}
    \item Prüfe ob Produkt bereits im Warenkorb:
    \begin{itemize}
        \item Wenn ja: Erhöhe Menge um 1
        \item Wenn nein: Füge neues Item hinzu
    \end{itemize}
    \item Session wird aktualisiert: \texttt{session["cart"] = cart}
    \item JSON-Response mit neuem Cart-Count wird zurückgegeben
    \item JavaScript aktualisiert UI (Cart-Counter wird inkrementiert)
\end{enumerate}

\section{Designprinzipien}
\label{sec:designprinzipien}

\subsection{Separation of Concerns}
\label{subsec:soc}

Das Projekt folgt dem Prinzip Separation of Concerns (SoC), das bedeutet jedes Modul hat eine klare, einzelne Verantwortung:

\begin{itemize}
    \item \textbf{Routes (app.py):} Nur HTTP-Handling, keine Business-Logic
    \item \textbf{Services (services/):} Nur Business-Logic, keine DB-Zugriffe direkt
    \item \textbf{Storage (storage/):} Nur Datenzugriff, keine Business-Logic
    \item \textbf{Utils (utils/):} Hilfs-Funktionen (Logging, Helpers)
    \item \textbf{Templates:} Nur Präsentation, keine Python-Logic
\end{itemize}

Diese Trennung ermöglicht:
\begin{itemize}
    \item \textbf{Testbarkeit:} Jedes Modul kann isoliert getestet werden
    \item \textbf{Wartbarkeit:} Änderungen an einer Schicht beeinflussen nicht andere Schichten
    \item \textbf{Wiederverwendbarkeit:} Modules können in anderen Projekten genutzt werden
\end{itemize}

\subsection{DRY (Don't Repeat Yourself)}
\label{subsec:dry}

Das DRY-Prinzip wird durchgehend angewandt:

\begin{itemize}
    \item \textbf{Hybrid Backend:} Einheitliche Daten-API statt separate SQLite- und CSV-Implementierungen
    \item \textbf{Template Inheritance:} \texttt{base.html} enthält gemeinsames Layout, alle anderen Templates erben davon
    \item \textbf{Utility Functions:} \texttt{get\_cart\_item\_count()}, \texttt{allowed\_file()} sind zentrale Helper
    \item \textbf{Audit-Logging:} Zentrale Logging-Klasse statt Log-Code überall verteilt
\end{itemize}

\section{Sicherheitsarchitektur}
\label{sec:sicherheit}

\subsection{Authentifizierung und Autorisierung}
\label{subsec:auth}

\paragraph{Authentifizierung (Authentication):}

Authentifizierung bestätigt die Identität eines Benutzers:

\begin{enumerate}
    \item \textbf{Registrierung:}
    \begin{itemize}
        \item Passwort wird mit \texttt{werkzeug.security.generate\_password\_hash()} gehasht
        \item Hash-Algorithmus: PBKDF2 mit Salt
        \item Beispiel Hash: \texttt{pbkdf2:sha256:260000\$...}
        \item Passwort wird \textbf{nicht} im Klartext gespeichert
    \end{itemize}
    
    \item \textbf{Login:}
    \begin{itemize}
        \item Benutzer gibt Email + Passwort ein
        \item System sucht User in Backend: \texttt{backend.get\_user\_by\_email(email)}
        \item Gespeicherter Hash wird mit Eingabe-Passwort verglichen: \texttt{check\_password\_hash(hash, password)}
        \item Bei Erfolg: Session wird erstellt mit User-ID, Name, Email, Role
    \end{itemize}
    
    \item \textbf{Session Management:}
    \begin{itemize}
        \item Flask-Sessions sind standardmäßig mit Secret Key signiert (HMAC)
        \item Session-Data wird Client-seitig als signiertes Cookie gespeichert
        \item Tampering wird automatisch erkannt
        \item Session-Timeout kann konfiguriert werden
    \end{itemize}
\end{enumerate}

\paragraph{Autorisierung (Authorization):}

Autorisierung bestimmt, was ein Benutzer darf:

\begin{itemize}
    \item \textbf{Role-Based Access Control (RBAC):} Zwei Rollen: \texttt{user} und \texttt{admin}
    
    \item \textbf{User-Berechtigungen:}
    \begin{itemize}
        \item Produkte browsen
        \item Warenkorb benutzen
        \item Bestellungen aufgeben
        \item Eigene Bestellungshistorie einsehen
        \item Eigene Daten exportieren
        \item Eigenes Konto löschen
    \end{itemize}
    
    \item \textbf{Admin-Berechtigungen:}
    \begin{itemize}
        \item Alle User-Berechtigungen PLUS
        \item Produkte erstellen/bearbeiten/löschen
        \item Alle Bestellungen einsehen
        \item Bestellungsstatus ändern
        \item Benutzer-Daten exportieren/löschen
    \end{itemize}
    
    \item \textbf{Implementierung:}
    \begin{itemize}
        \item Route-Level-Schutz: \texttt{if user.get("role") != "admin": return error}
        \item Admin-PIN bei Registrierung: \texttt{if role == "admin" and admin\_pin != ADMIN\_PIN: return error}
    \end{itemize}
\end{itemize}

\subsection{Schutz vor Angriffsmustern}
\label{subsec:schutz_angriffe}

\paragraph{CSRF (Cross-Site Request Forgery):}

Schutz vor Angriffen, bei denen Angreifer Benutzer zu unwollten Aktionen verleiten:

\begin{itemize}
    \item \textbf{Implementierung:} Flask-WTF bietet Token-basierte CSRF-Schutz
    \item \textbf{Token in Forms:} \texttt{\{\% csrf\_token() \%\}} wird in alle Formulare eingefügt
    \item \textbf{Validierung:} Flask validiert Token automatisch vor Request-Processing
\end{itemize}

\paragraph{XSS (Cross-Site Scripting):}

Schutz vor Angriffen durch JavaScript-Injection:

\begin{itemize}
    \item \textbf{Template Auto-Escaping:} Jinja2 escapt automatisch alle User-Eingaben in Templates
    \item \textbf{Sichere Filter:} \texttt{\{\{ user\_input | safe \}\}} nur für vertraute Daten
    \item \textbf{Content Security Policy:} Optional CSP-Header für zusätzlichen Schutz
\end{itemize}

\paragraph{SQL-Injection:}

Schutz vor SQL-Injection durch Datenbankabfragen:

\begin{itemize}
    \item \textbf{Parameterized Queries:} SQLite Backend nutzt Placeholder: \texttt{SELECT * FROM users WHERE email = ?}
    \item \textbf{ORM:} Würde vollständig vor SQL-Injection schützen (Empfehlung für Production)
    \item \textbf{Input Validation:} Zusätzlich werden Email-Adressen validiert (format-check)
\end{itemize}

\paragraph{Passwort-Sicherheit:}

\begin{itemize}
    \item \textbf{Hashing:} PBKDF2 mit mindestens 260.000 Iterationen (OWASP-Standard)
    \item \textbf{Salt:} Automatisches Salt durch werkzeug.security
    \item \textbf{Kein Plaintext:} Passwörter werden nie in Logs oder Backups sichtbar
    \item \textbf{Password Requirements:} Optional können Mindestanforderungen durchgesetzt werden (Mindestlänge, Komplexität)
\end{itemize}

\paragraph{Datei-Upload-Sicherheit:}

\begin{itemize}
    \item \textbf{Datei-Typ-Validierung:} Nur Bilder erlaubt (png, jpg, jpeg, gif)
    \item \textbf{Secure Filename:} \texttt{werkzeug.utils.secure\_filename()} verhindert Path Traversal
    \item \textbf{Größen-Limit:} Max. Dateigröße wird durchgesetzt
    \item \textbf{Virus-Scanning:} Optional integrierbar (ClamAV)
\end{itemize}

\paragraph{HTTPS/TLS:}

\begin{itemize}
    \item \textbf{Production:} HTTPS ist obligatorisch
    \item \textbf{Zertifikat:} Let's Encrypt oder kommerzielles Zertifikat
    \item \textbf{HSTS:} HTTP Strict Transport Security erzwingt HTTPS
\end{itemize}

\section{Performance- und Wartbarkeitsaspekte}
\label{sec:performance}

\subsection{Performance-Kriterien}
\label{subsec:performance_kriterien}

\paragraph{Seiten-Ladezeiten:}

Zielwerte für Produktionsumgebung:

\textit{(Tabelle 9 - siehe Anhang)}\newline

\paragraph{Optimierungsmaßnahmen:}

\begin{itemize}
    \item \textbf{Datenbank-Indizes:} Auf häufig abgefragte Felder (email, id, category)
    \item \textbf{Lazy Loading:} Bilder werden mit \texttt{loading="lazy"} nur bei Bedarf geladen
    \item \textbf{Pagination:} Große Produktlisten werden paginiert (z.B. 20 Produkte pro Seite)
    \item \textbf{Caching:} Flask kann einfaches HTTP-Caching nutzen (Cache-Control Header)
    \item \textbf{AJAX-Requests:} Warenkorb-Updates ohne Seite neu zu laden
    \item \textbf{Minification:} CSS und JavaScript werden minifiziert (optional)
\end{itemize}

\subsection{Wartbarkeit und Erweiterbarkeit}
\label{subsec:wartbarkeit}

\paragraph{Code-Struktur:}

\begin{itemize}
    \item \textbf{Modulare Architektur:} Backend und Frontend sind entkoppelt
    \item \textbf{Separation of Concerns:} Klare Verantwortlichkeiten
    \item \textbf{Dokumentation:} Docstrings in allen Funktionen
    \item \textbf{Konfigurierbar:} Settings via \texttt{config.py} und \texttt{.env}
\end{itemize}

\paragraph{Zukünftige Erweiterungen:}

Die Architektur ermöglicht folgende Erweiterungen ohne Major Refactoring:

\begin{itemize}
    \item \textbf{PostgreSQL-Migration:} Hybrid Backend kann auf PostgreSQL erweitert werden
    \item \textbf{Redis-Caching:} Session-Speicherung in Redis möglich
    \item \textbf{Asynchrone Tasks:} Celery für langlaufende Operationen (Email, Reports)
    \item \textbf{REST-API:} API-Endpoints können ausgebaut werden
    \item \textbf{Mobile App:} JSON-API ermöglicht Mobile-Clients
    \item \textbf{Search Engine:} Elasticsearch für erweiterte Produktsuche
    \item \textbf{Recommendations:} Machine-Learning für Produktempfehlungen
\end{itemize}

% ============================================================================
% KAPITEL 6: IMPLEMENTIERUNG UND MVP
% ============================================================================
\chapter{Implementierung und MVP}
\label{chap:implementierung}

\section{Projektplannung und Phasen}
\label{sec:projektplanung}

\subsection{Phasenmodell}
\label{subsec:phasenmodell}

% Platzhalter für Inhalt

\subsection{Zeitplanung und Meilensteine}
\label{subsec:zeitplanung}

% Platzhalter für Inhalt

\section{Kernfunktionalitäten des MVP}
\label{sec:mvp_funktionen}

\subsection{Benutzerverwaltung}
\label{subsec:user_mgmt}

% Platzhalter für Inhalt

\subsection{Produktkatalog}
\label{subsec:produktkatalog}

% Platzhalter für Inhalt

\subsection{Warenkorb und Checkout}
\label{subsec:warenkorb_checkout}

\paragraph{Warenkorb-Session:}

Der Warenkorb wird in der Flask-Session gespeichert (nicht in Datenbank):

\begin{verbatim}
session["cart"] = [
    {"product_id": "prod-1", "quantity": 2},
    {"product_id": "prod-5", "quantity": 1}
]
\end{verbatim}

\paragraph{Warenkorb-Operationen:}

\begin{itemize}
    \item \textbf{In Warenkorb legen:} AJAX-POST zu \texttt{/add-to-cart}
    \item \textbf{Menge ändern:} POST zu \texttt{/update-cart/<product\_id>}
    \item \textbf{Aus Warenkorb entfernen:} POST zu \texttt{/remove-from-cart/<product\_id>}
    \item \textbf{Warenkorb anzeigen:} GET \texttt{/cart} — Abrufen von Produktdetails und Gesamtpreis
\end{itemize}

\paragraph{Checkout-Prozess:}

Der Checkout folgt diesem Ablauf:

\begin{enumerate}
    \item Benutzer navigiert zu \texttt{/checkout}
    \item Checkout-Template zeigt Artikel-Zusammenfassung
    \item Benutzer füllt Kundendaten aus (Name, Adresse)
    \item Benutzer wählt Zahlungsart (Stripe oder PayPal)
    \item POST zu \texttt{/checkout/create}
    \item Backend erstellt vorläufige Bestellung mit Status \texttt{pending}
    \item Abhängig von Zahlungsart wird zu externem Payment Gateway weitergeleitet
    \item Nach erfolgreicher Zahlung: Benutzer zu \texttt{/checkout/success}
    \item Status wird auf \texttt{paid} aktualisiert
    \item Bestellbestätigungsseite anzeigen
    \item Warenkorb leeren
\end{enumerate}

\subsection{Bestellungsverwaltung}
\label{subsec:bestellungen}

\paragraph{Bestellungs-Lifecycle:}

Bestellungen durchlaufen folgende Status-Übergänge:

\begin{itemize}
    \item \texttt{pending} → \texttt{paid} (nach erfolgreicher Zahlung)
    \item \texttt{paid} → \texttt{in\_bearbeitung} (Lagerprüfung)
    \item \texttt{in\_bearbeitung} → \texttt{vorbereitung\_transport} (Verpackung)
    \item \texttt{vorbereitung\_transport} → \texttt{abgeschickt} (Versand)
    \item \texttt{abgeschickt} → \texttt{zugestellt} (Lieferung)
\end{itemize}

\paragraph{Benutzer-Seite:}

Benutzer sehen nur ihre eigenen Bestellungen mit Details wie Bestellnummer, Datum, Betrag und Status.

\paragraph{Admin-Seite:}

Administratoren sehen alle Bestellungen und können Status ändern oder Bestellungen löschen.

\subsection{Admin-Interface}
\label{subsec:admin_interface}

Das Admin-Interface ermöglicht Verwaltung von Produkten und Bestellungen.

\paragraph{Produkt-Verwaltung:}

\begin{itemize}
    \item Alle Produkte in Tabelle anzeigen
    \item Neues Produkt mit Formular hinzufügen
    \item Produkte bearbeiten oder löschen
    \item Bilder hochladen und verwalten
\end{itemize}

\paragraph{Bestellungs-Verwaltung:}

\begin{itemize}
    \item Alle Bestellungen einsehen
    \item Bestellungs-Status ändern
    \item Bestellungen mit Bestätigung löschen
\end{itemize}

\section{Implementierte DSGVO-Funktionen}
\label{sec:dsgvo_implementierung}

DSGVO-Compliance war ein Kern-Ziel des Projekts mit folgenden implementierten Funktionalitäten.

\subsection{Datenexport und Löschung}
\label{subsec:export_loeschung}

\paragraph{Datenexport (Art. 15 DSGVO):}

Benutzer können unter \texttt{/gdpr/data-export} ihre Daten exportieren:

\begin{itemize}
    \item Vollständiges Benutzerprofil
    \item Alle Bestellungen und Bestellpositionen
    \item Alle Einwilligungen (Consents)
    \item Audit-Logs der eigenen Aktivitäten
    \item Export als JSON-Datei herunterladbar
\end{itemize}

\paragraph{Kontolöschung (Art. 17 DSGVO):}

Benutzer können ihr Konto löschen mit automatischer Anonymisierung:

\begin{itemize}
    \item Benutzerprofil wird gelöscht
    \item Bestellungen werden anonymisiert (Namen entfernt, aber Daten bleibt)
    \item Einwilligungen werden gelöscht
    \item Audit-Logs werden dokumentiert
\end{itemize}

\subsection{Audit-Logging und Datenschutzkonformität}
\label{subsec:audit_logging}

Ein zentrales Audit-Logging System protokolliert alle kritischen Operationen:

\begin{itemize}
    \item USER\_REGISTRATION — Neue Registrierung
    \item USER\_LOGIN — Benutzer angemeldet
    \item ORDER\_CREATED — Bestellung aufgegeben
    \item PAYMENT\_COMPLETED — Zahlung erfolgreich
    \item USER\_DATA\_EXPORT — Datenexport
    \item USER\_DATA\_DELETED — Konto gelöscht
    \item PRODUCT\_CREATED/UPDATED/DELETED — Admin-Aktionen
\end{itemize}

Jeder Audit-Log-Eintrag enthält: Event-Typ, Benutzer-ID, Timestamp, IP-Adresse, Details.

Die Logs werden in \texttt{data/logs/audit\_log.csv} gespeichert.

\section{Herausforderungen und Lösungen}
\label{sec:herausforderungen}

Während der Implementierung traten verschiedene Herausforderungen auf:

\paragraph{Herausforderung 1: CSV zu SQLite Migration}

\textbf{Problem:} Initiale Daten in CSV, System soll SQLite nutzen.

\textbf{Lösung:} Migration-Script \texttt{migrate\_csv\_to\_sqlite.py} liest CSV, transformiert Daten, schreibt zu SQLite.

\paragraph{Herausforderung 2: Zahlungsintegration}

\textbf{Problem:} Stripe und PayPal haben unterschiedliche APIs.

\textbf{Lösung:} Abstraktions-Layer (Services) mit uniformer Schnittstelle für beide Provider.

\paragraph{Herausforderung 3: DSGVO-Compliance bei Bestellungen}

\textbf{Problem:} Benutzer löschen, aber Bestellungen müssen erhalten bleiben (Steuern).

\textbf{Lösung:} Anonymisierungslogik mit Markierung gelöschter Benutzer.

\paragraph{Herausforderung 4: Responsive Design}

\textbf{Problem:} Shop muss auf Mobile, Tablet, Desktop funktionieren.

\textbf{Lösung:} CSS Media Queries und Flexbox für responsive Layouts.

% ============================================================================
% KAPITEL 7: TESTING UND QUALITÄTSSICHERUNG
% ============================================================================
\chapter{Testing und Qualitätssicherung}
\label{chap:testing}

\section{Teststrategien}
\label{sec:teststrategien}

Das Projekt wurde als Solo-Projekt vollständig vom Autor entwickelt und getestet. Die Qualitätssicherung erfolgte hauptsächlich durch systematisches manuelles Testing mit begleitenden automatisierten Unit-Tests.

\subsection{Manuelles Testing}
\label{subsec:manual_testing}

Das manuelle Testen war die primäre Teststrategie mit folgenden Schwerpunkten:

\paragraph{Benutzer-Workflows (End-to-End Testing):}

Alle kritischen Benutzer-Journeys wurden manuell durchgetestet:

\begin{enumerate}
    \item \textbf{Registrierungs-Workflow:}
    \begin{itemize}
        \item Registrierung mit gültigen Daten
        \item Registrierung mit ungültiger Email (Format-Validierung)
        \item Registrierung mit bereits existierender Email (Duplikat-Check)
        \item Passwort-Validierung (Längde, Sonderzeichen)
        \item Einwilligungen erforderlich (Privacy, AGB)
        \item Session wird nach Registrierung gesetzt
    \end{itemize}
    
    \item \textbf{Login-Workflow:}
    \begin{itemize}
        \item Login mit korrekten Credentials
        \item Login mit falscher Email
        \item Login mit falschem Passwort
        \item Session wird korrekt gespeichert
        \item User-Info (Name, Role) ist in Session korrekt
    \end{itemize}
    
    \item \textbf{Produktbrowsing:}
    \begin{itemize}
        \item Startseite lädt alle Produkte
        \item Suchfunktion filtert nach Produktname
        \item Kategorie-Filter funktioniert
        \item Preis-Filter (Min/Max) funktioniert
        \item Kombination von Filtern (Kategorie + Preis)
        \item Produktdetail-Seite laden
    \end{itemize}
    
    \item \textbf{Warenkorb-Workflow:}
    \begin{itemize}
        \item Produkt zu Warenkorb hinzufügen (AJAX)
        \item Warenkorb-Counter wird aktualisiert
        \item Warenkorb-Seite zeigt alle Items
        \item Menge ändern funktioniert
        \item Item aus Warenkorb entfernen funktioniert
        \item Gesamtpreis wird korrekt berechnet
    \end{itemize}
    
    \item \textbf{Checkout-Workflow (Stripe-Flow):}
    \begin{itemize}
        \item Checkout-Seite zeigt Warenkorb-Zusammenfassung
        \item Kundendaten-Formular funktioniert
        \item Stripe Payment Option auswählbar
        \item Redirect zu Stripe Checkout (simuliert)
        \item Nach erfolgreicher Zahlung: Bestellbestätigung
        \item Bestellnummer wird angezeigt
        \item Warenkorb wird geleert
    \end{itemize}
    
    \item \textbf{Bestellungshistorie:}
    \begin{itemize}
        \item Benutzer sieht nur ihre eigenen Bestellungen
        \item Bestelldetails sind korrekt
        \item Bestellungs-Status ist aktuell
    \end{itemize}
    
    \item \textbf{Admin-Funktionalitäten:}
    \begin{itemize}
        \item Admin-Login mit PIN-Validierung
        \item Produkt erstellen funktioniert
        \item Produkt bearbeiten funktioniert
        \item Produktbilder hochladen funktioniert
        \item Produkt löschen funktioniert
        \item Bestellungen verwalten funktioniert
        \item Status-Updates funktionieren
    \end{itemize}
    
    \item \textbf{DSGVO-Funktionalitäten:}
    \begin{itemize}
        \item Datenschutzerklärung ist zugänglich
        \item Impressum ist korrekt
        \item AGB sind korrekt
        \item Datenexport funktioniert (JSON Download)
        \item Kontolöschung funktioniert (Anonymisierung)
        \item Audit-Logs werden erstellt
    \end{itemize}
\end{enumerate}

\paragraph{Browser- und Geräte-Kompatibilität:}

Manuelles Testing auf verschiedenen Plattformen:

\begin{itemize}
    \item \textbf{Desktop:} Chrome, Firefox, Edge (Windows)
    \item \textbf{Responsive Design:} Chrome DevTools für Mobile-Simulation (320px, 768px, 1024px)
    \item \textbf{Bildschirmgrößen:} 
    \begin{itemize}
        \item Mobile (< 768px): Hamburger-Menü, Stack-Layout
        \item Tablet (768-1024px): Hybrid-Layout
        \item Desktop (> 1024px): Full-Grid
    \end{itemize}
\end{itemize}

\subsection{Automatisierte Unit-Tests}
\label{subsec:unit_tests}

Zusätzlich zu manuellem Testing wurden ausgewählte kritische Funktionen durch automatisierte Unit-Tests abgedeckt. Diese Tests sind in \texttt{tests/} vorhanden:

\paragraph{Storage-Tests (test\_storage.py):}

\begin{itemize}
    \item \textbf{CSV Backend Tests:}
    \begin{itemize}
        \item CSV lesen funktioniert
        \item CSV schreiben funktioniert
        \item Neue Einträge werden korrekt hinzugefügt
    \end{itemize}
    
    \item \textbf{SQLite Backend Tests:}
    \begin{itemize}
        \item Datenbankverbindung funktioniert
        \item INSERT-Operationen funktionieren
        \item SELECT-Abfragen funktionieren
        \item UPDATE/DELETE funktionieren
    \end{itemize}
\end{itemize}

Die automatisierten Tests können mit \texttt{pytest} ausgeführt werden:

\begin{verbatim}
pytest tests/test_storage.py -v
\end{verbatim}

\paragraph{Limitierungen der automatisierten Tests:}

Aufgrund des Solo-Entwicklungs-Ansatzes liegt der Fokus auf manuellem Testing aus folgenden Gründen:

\begin{itemize}
    \item \textbf{Zeiteffizienz:} Manuelles E2E-Testing ist für MVP schneller als umfassende Automatisierung
    \item \textbf{UI-Testing:} Flask-Sessions, JavaScript-Interaktionen sind schwer zu automatisieren
    \item \textbf{Externe APIs:} Stripe/PayPal Integration ist schwer zu testen (Sandbox-Simulationen erforderlich)
    \item \textbf{Priorisierung:} Funktionalität wurde über 100\% Test-Coverage priorisiert
\end{itemize}

\section{Qualitätskriterien und Validierung}
\label{sec:qualitaetskriterien}

\subsection{Funktionalitäts-Validierung}
\label{subsec:funktionalitaet}

Alle beschriebenen Anforderungen aus Kapitel 1 wurden validiert:

\textit{(Tabelle 10 - siehe Anhang)}\newline

\subsection{Sicherheits-Validierung}
\label{subsec:sicherheitstests}

Sicherheitskritische Aspekte wurden manuell validiert:

\paragraph{Authentifizierung \& Session-Sicherheit:}

\begin{itemize}
    \item \textbf{Passwort-Hashing:} Passwords sind mit PBKDF2 gehasht, nicht im Klartext
    \item \textbf{Session-Signing:} Flask-Sessions sind mit Secret Key signiert
    \item \textbf{Logout:} Session wird korrekt gelöscht nach Logout
    \item \textbf{Admin-Protection:} Routes mit Admin-Check schützen restricted Access
\end{itemize}

\paragraph{Input-Validierung:}

\begin{itemize}
    \item \textbf{Email-Validierung:} Email-Format wird geprüft (Basic Format Check)
    \item \textbf{SQL-Injection:} Parameterized Queries verhindern SQL-Injection
    \item \textbf{XSS-Schutz:} Jinja2 Template Auto-Escaping schützt vor XSS
    \item \textbf{CSRF-Protection:} CSRF-Tokens in Formularen
\end{itemize}

\paragraph{Datenschutz \& Compliance:}

\begin{itemize}
    \item \textbf{Keine Zahlungsdaten:} Keine Kreditkartendaten werden gespeichert (Delegiert an Stripe/PayPal)
    \item \textbf{Audit-Logs:} Alle kritischen Operationen werden protokolliert
    \item \textbf{Datenminimierung:} Nur notwendige Daten werden erfasst
    \item \textbf{Datenlöschung:} Benutzer können ihre Daten löschen
\end{itemize}

\paragraph{Datei-Upload-Sicherheit:}

\begin{itemize}
    \item \textbf{Datei-Typ-Validierung:} Nur Bilder erlaubt (png, jpg, jpeg, gif)
    \item \textbf{Secure Filename:} Pfad-Traversal wird verhindert
    \item \textbf{Upload-Folder:} Isoliert in \texttt{static/uploads/}
\end{itemize}

\subsection{Performance-Validierung}
\label{subsec:performance_tests}

Performance wurde manuell überwacht während des Testing:

\paragraph{Ladezeiten:}

\begin{itemize}
    \item \textbf{Startseite:} ~ 0,5–1s (mit 10 Produkten)
    \item \textbf{Produktdetail:} ~ 0,2–0,5s
    \item \textbf{Warenkorb:} ~ 0,2–0,3s
    \item \textbf{Checkout:} ~ 0,5–1s
    \item \textbf{Admin-Interface:} ~ 0,5–1s
\end{itemize}

Diese Zeiten entsprechen den Zielwerten aus Kapitel 5 und sind für MVP angemessen.

\paragraph{Skalierbarkeit:}

Das System wurde mit folgenden Datenmengen getestet:

\begin{itemize}
    \item \textbf{Produkte:} 50 (Realistic für MVP)
    \item \textbf{Benutzer:} 20 (Testuser)
    \item \textbf{Bestellungen:} 15 (Test-Bestellungen)
\end{itemize}

Mit diesen Mengen funktioniert das System stabil. Für größere Datenmengen (1000+ Produkte) wären Optimierungen nötig (Indizes, Pagination, Caching).

\section{Usability und Benutzer-Experience}
\label{sec:usability}

\subsection{Benutzer-Feedback und Iterationen}
\label{subsec:user_feedback}

Als Solo-Projekt lag die QA vollständig beim Entwickler mit folgenden Validierungskriterien:

\paragraph{Navigation \& Verständlichkeit:}

\begin{itemize}
    \item Sind alle Funktionen leicht zu finden?
    \item Sind Menüs logisch strukturiert?
    \item Sind Call-to-Action Buttons klar erkennbar?
    \item Ist die Seiten-Hierarchie nachvollziehbar?
\end{itemize}

\paragraph{Fehlerbehandlung:}

\begin{itemize}
    \item Sind Fehlermeldungen verständlich?
    \item Werden Input-Fehler klar erklärt?
    \item Gibt es Hinweise auf nächste Schritte nach Fehlern?
    \item Sind 404/500 Fehler angemessen gestaltet?
\end{itemize}

\paragraph{Responsive Design:}

\begin{itemize}
    \item Mobile View: Touch-freundliche Buttons, lesbare Fonts
    \item Tablet View: Optimale Layoutverteilung
    \item Desktop View: Volle Funktionalität sichtbar
    \item Keine Horizontal-Scrolling nötig (außer erwünscht)
\end{itemize}

\paragraph{Accessibility:}

\begin{itemize}
    \item Alt-Text für Produktbilder vorhanden
    \item Klare Kontraste zwischen Text und Background
    \item Keyboard-Navigation funktioniert
    \item Buttons haben aussagekräftige Labels
\end{itemize}

\subsection{Browser-Testing Ergebnisse}
\label{subsec:browser_testing}

Manuelle Tests in verschiedenen Browsern zeigten folgende Ergebnisse:

\textit{(Tabelle 11 - siehe Anhang)}\newline

% ============================================================================
% KAPITEL 8: KRITISCHE REFLEXION
% ============================================================================
\chapter{Kritische Reflexion}
\label{chap:reflexion}

Diese Projektarbeit war eine umfassende Lernreise mit wertvoollen Erkenntnissen. Die nachfolgende kritische Reflexion behandelt erfolgreiche Aspekte, Herausforderungen und Lernpunkte.

\section{Erfolgreiche Aspekte}
\label{sec:erfolg}

Mehrere Aspekte des Projekts können als besonders erfolgreich bewertet werden:

\subsection{Hybrid-Backend-Architektur}
\label{subsec:erfolg_hybrid}

Die Implementierung des Hybrid-Backends (SQLite mit CSV-Fallback) war eine gelungene Design-Entscheidung:

\begin{itemize}
    \item \textbf{Redundanz:} Dual-Speicherung bietet robustes Fallback bei Datenbankfehlern
    \item \textbf{Flexibilität:} Migration von CSV zu SQLite war möglich ohne Datenverlust
    \item \textbf{Debugging:} CSV-Dateien sind human-readable und unterstützen Development
    \item \textbf{Lernergebnis:} Vertiefte Verständnis für Datenspeicher-Architektur
\end{itemize}

Diese Lösung hätte auch Production-Wert für Systeme mit hohen Verfügbarkeitsanforderungen.

\subsection{DSGVO-Compliance Implementation}
\label{subsec:erfolg_dsgvo}

Die vollständige DSGVO-Implementierung war ambitioniert und erfolgreich:

\begin{itemize}
    \item \textbf{Audit-Logging:} Zentrales System dokumentiert alle kritischen Operationen
    \item \textbf{Datenexport:} Benutzer können ihre Daten jederzeit exportieren (Art. 15)
    \item \textbf{Rechtlöschung:} Kontolöschung funktioniert mit Anonymisierung (Art. 17)
    \item \textbf{Konformität:} Shop erfüllt realistische DSGVO-Anforderungen
\end{itemize}

Dies unterscheidet den Shop von typischen MVP-Implementierungen und demonstriert Datenschutz-Bewusstsein.

\subsection{Modular und wartbarer Code}
\label{subsec:erfolg_code}

Die Trennung in mehrere Module ermöglichte gute Wartbarkeit:

\begin{itemize}
    \item \textbf{Storage Layer:} CSV, SQLite, Hybrid Backend sind austauschbar
    \item \textbf{Services Layer:} Checkout-Logik ist unabhängig von Routes
    \item \textbf{Utils:} Logging und Helper sind zentral und wiederverwendbar
    \item \textbf{Templates:} Template Inheritance reduziert Code-Duplikation
\end{itemize}

Zukünftige Entwickler können leicht Module austauschen oder erweitern.

\subsection{Benutzerfreundliche UI}
\label{subsec:erfolg_ui}

Das Frontend wurde intuitiv und responsive entwickelt:

\begin{itemize}
    \item \textbf{Produktgrid:} Übersichtlich mit Such-/Filterfunktion
    \item \textbf{Checkout-Prozess:} Klar strukturiert und einfach zu verstehen
    \item \textbf{Mobile-Optimierung:} Shop funktioniert auf allen Geräten
    \item \textbf{Fehlermeldungen:} Klare Kommunikation bei Problemen
\end{itemize}

Usability-Tests zeigten, dass Nutzer den Shop ohne Anleitung verstehen.

\section{Herausforderungen und Lernpunkte}
\label{sec:lernpunkte}

\subsection{Zahlungsintegration Komplexität}
\label{subsec:heraus_zahlung}

\textbf{Herausforderung:} Stripe und PayPal zu integrieren war deutlich komplexer als initial geplant.

\textbf{Lernpunkt:} 
\begin{itemize}
    \item Externe APIs erfordern tieferes Verständnis (OAuth, Webhooks, Error Handling)
    \item Test-Umgebungen (Sandbox) sind essentiell, aber auch eigene Tücken
    \item Webhook-Handling für asynchrone Payment-Bestätigungen ist komplex
\end{itemize}

\textbf{Was ich anders machen würde:} Früher mit Sandbox-Testing beginnen, Mock-Payment-Flows verwenden.

\subsection{Session vs. Datenbank für Warenkorb}
\label{subsec:heraus_warenkorb}

\textbf{Herausforderung:} Warenkorb in Session speichern ist einfach, aber nicht persistent.

\textbf{Gewählter Ansatz:} Session-basiert für MVP ist pragmatisch.

\textbf{Limitation:} 
\begin{itemize}
    \item Warenkorb geht verloren bei Browser-Neustart (ohne Cookies)
    \item Nicht möglich, Warenkorb zwischen Devices zu synchronisieren
    \item Für Production: Datenbank-Speicherung empfohlen
\end{itemize}

\textbf{Lernpunkt:} Pragmatische Tradeoffs sind wichtig für MVP, aber müssen dokumentiert sein.

\subsection{CSV zu SQLite Migration}
\label{subsec:heraus_migration}

\textbf{Herausforderung:} Testdaten in CSV zu migrieren erforderte sorgfältige Datentyp-Konvertierung.

\textbf{Lösung:} Migrationsscript mit Validierung und Error-Handling.

\textbf{Lernpunkt:} 
\begin{itemize}
    \item Datentyp-Konsistenz ist kritisch (String vs. Int vs. Float)
    \item JSON-Felder in SQLite erfordern Serialisierung
    \item Idempotenz wichtig (Script kann mehrfach laufen)
\end{itemize}

\subsection{Testing-Strategie}
\label{subsec:heraus_testing}

\textbf{Herausforderung:} Mit limitierter Zeit musste zwischen automatisierten und manuellen Tests abgewogen werden.

\textbf{Gewählter Ansatz:} Hauptsächlich manuelles E2E-Testing mit ausgewählten Unit-Tests.

\textbf{Pro:}
\begin{itemize}
    \item Schneller für MVP-Development
    \item Besseres Verständnis für User-Flows
    \item Einfaches Debugging während Testing
\end{itemize}

\textbf{Contra:}
\begin{itemize}
    \item Nicht skalierbar bei vielen Code-Änderungen
    \item Regression-Testing manuell zeitintensiv
    \item Subjektiv (kann Fehler übersehen)
\end{itemize}

\textbf{Lernpunkt:} Für Production wäre umfassendere Test-Automatisierung notwendig.

\subsection{Fehlerbehandlung}
\label{subsec:heraus_fehler}

\textbf{Herausforderung:} Robuste Fehlerbehandlung für externe APIs (Stripe, PayPal) ist komplex.

\textbf{Was gut funktioniert:}
\begin{itemize}
    \item Fallback auf CSV bei SQLite-Fehlern
    \item Aussagekräftige Error-Messages für Benutzer
    \item Logging von Exceptions für Debugging
\end{itemize}

\textbf{Was verbessert werden könnte:}
\begin{itemize}
    \item Retry-Logik für transiente Fehler
    \item Circuit-Breaker Pattern für externe APIs
    \item Bessere Recovery-Strategien
\end{itemize}

\section{Alternative Ansätze}
\label{sec:alternativen}

\subsection{ORM statt Raw SQL}
\label{subsec:alt_orm}

\textbf{Gewählter Ansatz:} SQLite mit raw SQL via sqlite3.

\textbf{Alternative:} SQLAlchemy ORM

\textbf{Vergleich:}

\textit{(Tabelle 12 - siehe Anhang)}\newline

\textbf{Fazit:} Für MVP war raw SQL pragmatisch. Für Production würde SQLAlchemy + Alembic empfohlen.

\subsection{Frontend-Framework}
\label{subsec:alt_frontend}

\textbf{Gewählter Ansatz:} Vanilla HTML, CSS, JavaScript (Jinja2 Templates).

\textbf{Alternativen:} React, Vue.js, Alpine.js

\textbf{Begründung für Vanilla:}
\begin{itemize}
    \item Minimale Dependencies (schneller Load)
    \item Einfaches Debugging
    \item Ausreichend für MVP-Komplexität
\end{itemize}

\textbf{Wann würde ich Framework nutzen:}
\begin{itemize}
    \item Single-Page Application (SPA) erforderlich
    \item Komplexe Frontend-State-Management
    \item Team-Development mit Standards
\end{itemize}

Für dieses Projekt war die Wahl richtig.

\subsection{Zahlungsabwicklung Delegation}
\label{subsec:alt_zahlung}

\textbf{Gewählter Ansatz:} Vollständige Delegierung an Stripe und PayPal.

\textbf{Alternative:} Eigene Payment Processing (nicht empfohlen!)

\textbf{Gründe für Delegation:}
\begin{itemize}
    \item PCI-DSS Compliance extrem komplex ohne Provider
    \item Sicherheit ist kritischer als Kosteneffizienz
    \item Provider handeln Regulatory Anforderungen
\end{itemize}

Dies war eine richtige Entscheidung.

\section{Skalierbarkeit und zukünftige Erweiterungen}
\label{sec:skalierbarkeit}

\subsection{Identifizierte Bottlenecks}
\label{subsec:bottlenecks}

Bei Skalierung würden folgende Bottlenecks auftreten:

\paragraph{Datenbank-Bottleneck:}

\begin{itemize}
    \item \textbf{Problem:} SQLite hat Concurrency-Limitierungen (nur 1 Writer gleichzeitig)
    \item \textbf{Symptom:} Bei vielen gleichzeitigen Bestellungen (> 100/min) würden Timeouts auftreten
    \item \textbf{Lösung:} Migration zu PostgreSQL oder MySQL
\end{itemize}

\paragraph{Session-Bottleneck:}

\begin{itemize}
    \item \textbf{Problem:} Flask-Sessions in Memory speichern ist nicht skalierbar
    \item \textbf{Symptom:} Bei Server-Restart gehen alle Sessions verloren
    \item \textbf{Lösung:} Redis für Session-Storage
\end{itemize}

\paragraph{Warenkorb-Bottleneck:}

\begin{itemize}
    \item \textbf{Problem:} Session-basierter Warenkorb funktioniert nur auf einem Server
    \item \textbf{Symptom:} Load-Balancing unmöglich
    \item \textbf{Lösung:} Warenkorb in Redis oder Datenbank verschieben
\end{itemize}

\paragraph{Bilder-Bottleneck:}

\begin{itemize}
    \item \textbf{Problem:} Alle Bilder im selben Filesystem
    \item \textbf{Symptom:} Filesystem wird Bottleneck bei vielen Produkten
    \item \textbf{Lösung:} Cloud Storage (AWS S3, Azure Blob, etc.)
\end{itemize}

\subsection{Verbesserungspotenziale}
\label{subsec:verbesserungen}

Folgende Verbesserungen würden die Qualität erhöhen:

\paragraph{Höchste Priorität:}

\begin{itemize}
    \item \textbf{Automated Testing:} Unit-Tests und Integrationstests for critical paths
    \item \textbf{Database Indizes:} Auf häufig queried Feldern (email, product\_id, status)
    \item \textbf{Caching:} HTTP-Caching für Produktseiten, Redis für Sessions
    \item \textbf{Monitoring:} Logging, Error Tracking (Sentry), Performance Monitoring
\end{itemize}

\paragraph{Mittlere Priorität:}

\begin{itemize}
    \item \textbf{Search:} Elasticsearch für Volltextsuche statt simple LIKE-Queries
    \item \textbf{Pagination:} Produktlisten mit Pagination (derzeit alle auf eine Seite)
    \item \textbf{Recommendations:} Personalisierte Produktempfehlungen
    \item \textbf{Reviews:} Benutzer können Produkte bewerten
\end{itemize}

\paragraph{Niedrigere Priorität (Nice-to-Have):}

\begin{itemize}
    \item \textbf{Multi-Language:} i18n Support
    \item \textbf{Wishlist:} Benutzer können Produkte merken
    \item \textbf{Newsletter:} Marketing Automation
    \item \textbf{Analytics:} Google Analytics, Conversion Tracking
\end{itemize}

\subsection{Mögliche zukünftige Features}
\label{subsec:future_features}

Das MVP bietet eine gute Basis für Erweiterungen:

\begin{itemize}
    \item \textbf{Inventory Management:} Echtzeit-Lagerbestandsverwaltung mit Low-Stock Alerts
    \item \textbf{Shipping Integration:} DHL/DPD/UPS API für automatische Versand-Labels
    \item \textbf{Subscription/Recurring:} Wiederholte Bestellungen
    \item \textbf{Affiliate Program:} Partner können Links teilen und kommissionen erhalten
    \item \textbf{API:} RESTful API für Mobile-Apps und Drittanbieter
    \item \textbf{Admin Dashboard:} Grafiken für Verkäufe, Top-Produkte, Kundenanalyse
    \item \textbf{Email Notifications:} Bestellbestätigung, Versand-Updates, Promotion-Emails
    \item \textbf{Advanced Search:} Faceted Search mit Filters
\end{itemize}

Die modulare Architektur ermöglicht diese Erweiterungen ohne Major Refactoring.

\section{Ressourcennutzung und Effizienz}
\label{sec:ressourcen}

\subsection{Zeitaufwand}
\label{subsec:zeitaufwand}

Das Projekt erforderte folgende Ressourcen:

\textit{(Tabelle 13 - siehe Anhang)}\newline

Dies ist ein realistischer Aufwand für einen funktionsfähigen MVP mit guter Qualität.

\subsection{Externe Ressourcen}
\label{subsec:externe_ressourcen}

Folgende externe Ressourcen wurden genutzt:

\begin{itemize}
    \item \textbf{Dokumentation:} Flask, SQLite, Stripe, PayPal Official Docs
    \item \textbf{Frameworks:} Flask (Web), pandas (Data), werkzeug (Security)
    \item \textbf{Zahlungsdienste:} Stripe Sandbox, PayPal Sandbox (kostenlos)
    \item \textbf{Versionskontrolle:} Git \& GitHub für Collaboration/Backup
    \item \textbf{Code-Editor:} VS Code mit Python Extensions
\end{itemize}

Alle verwendeten Tools sind Open-Source oder kostenlos, keine Lizenzkosten.

\subsection{Was hätte besser sein können}
\label{subsec:besser}

\paragraph{Zeitmanagement:}

\begin{itemize}
    \item \textbf{Real:} Zahlungsintegration dauerte länger als erwartet
    \item \textbf{Verbesserung:} Buffer für unerwartete Komplexität einplanen
\end{itemize}

\paragraph{Dokumentation:}

\begin{itemize}
    \item \textbf{Real:} Dokumentation wurde erst am Ende geschrieben
    \item \textbf{Verbesserung:} Dokumentation parallel entwickeln (Living Documentation)
\end{itemize}

\paragraph{Requirements:}

\begin{itemize}
    \item \textbf{Real:} Einige Features (z.B. CSV Migration) waren nicht initial geplant
    \item \textbf{Verbesserung:} Requirements freezing nach 2 Wochen, dann Scope Lock
\end{itemize}

% ============================================================================
% KAPITEL 9: FAZIT UND AUSBLICK
% ============================================================================
\chapter{Fazit und Ausblick}
\label{chap:fazit}

\section{Zusammenfassung der Ergebnisse}
\label{sec:zusammenfassung}

Das vorliegende Projekt hat erfolgreich einen funktionsfähigen, sicheren und datenschutzkonformen Onlineshop als Minimum Viable Product (MVP) entwickelt. Die Implementierung zeigt praktische Anwendung von Web-Engineering-Grundlagen mit modernen Technologien und Best Practices.

\paragraph{Kern-Deliverables:}

\begin{itemize}
    \item \textbf{Funktionsfähiger Webshop:} Mit vollständigem Produktkatalog, Warenkorb, Checkout und Bestellungsverwaltung
    \item \textbf{Sichere Authentifizierung:} Mit Password Hashing, Session Management und Role-Based Access Control
    \item \textbf{Zahlungsintegration:} Mit Stripe und PayPal für realistische Checkout-Prozesse
    \item \textbf{DSGVO-Compliance:} Mit Audit-Logging, Datenexport und Löschungsfunktionalitäten
    \item \textbf{Modulare Architektur:} Mit klarer Separation of Concerns und Hybrid-Backend-Design
    \item \textbf{Responsive UI:} Mit Unterstützung für Desktop, Tablet und Mobile
\end{itemize}

Die Implementierung demonstriert tiefgehendes Verständnis für Anforderungsanalyse, Softwarearchitektur, Sicherheit und Compliance.

\section{Erreichte Ziele}
\label{sec:ziele_erreicht}

Alle in der Aufgabenstellung definierten Ziele wurden erreicht:

\textit{(Tabelle 14 - siehe Anhang)}\newline

Zusätzlich wurden Qualitätsaspekte wie umfassendes Testing, Sicherheitsvalidierung und Usability-Evaluation durchgeführt.

\section{Erkenntnisse und Schlussfolgerungen}
\label{sec:erkenntnisse}

Aus diesem Projekt ergeben sich mehrere wichtige Erkenntnisse für zukünftige Web-Engineering-Projekte:

\paragraph{Erkenntnis 1: Architektur ist wichtiger als Technologie}

Die Wahl zwischen SQL vs. NoSQL, Framework A vs. Framework B ist weniger kritisch als eine gute, modulare Architektur. Das Hybrid-Backend hätte in jeder Technologie umgesetzt werden können — die Idee ist technologie-unabhängig und bietet realen Wert.

\paragraph{Erkenntnis 2: DSGVO ist nicht optional, sondern notwendig}

Datenschutz sollte nicht als Nachgedanke implementiert werden, sondern von Anfang an in die Architektur integriert sein. Ein früher DSGVO-Design ermöglicht:
\begin{itemize}
    \item Audit-Trails ohne großen Overhead
    \item Datenminimierung als Designprinzip
    \item Einfacheres Compliance-Auditing
\end{itemize}

\paragraph{Erkenntniss 3: Pragmatismus schlägt Perfektionismus beim MVP}

Das MVP beweist Konzept mit ~40 Features statt 400. Eine Session-basierte Warenkorb-Lösung funktioniert initial besser als eine Production-Grade Implementierung. Später kann optimiert werden.

\paragraph{Erkenntnisse 4: Testing-Strategie hängt vom Kontext ab}

Für MVP ist manuelles E2E-Testing oft effizienter als automatisierte Unit-Tests. Aber für langfristige Wartung ist Automatisierung essentiell. Eine Hybrid-Strategie ist optimal.

\paragraph{Erkenntnisse 5: Externe Services delegieren wo möglich}

Zahlungen an Stripe/PayPal zu delegieren ist nicht Laziness, sondern best practice. Es reduziert Sicherheitsrisiko, regulatorische Komplexität und Betriebsaufwand. ``Build vs. Buy`` sollte zugunsten von ``Buy'' entschieden werden für kritische Systeme.

\section{Ausblick auf zukünftige Entwicklungen}
\label{sec:ausblick}

\subsection{Kurz-Fristig (1-3 Monate)}
\label{subsec:kurzfristig}

Folgende Verbesserungen sollten in den nächsten Monaten priorisiert werden:

\begin{enumerate}
    \item \textbf{Datenbankindizes:} Performance optimieren für größere Datenmengen
    \item \textbf{Automated Testing:} Unit-Tests für Core Business Logic aufbauen
    \item \textbf{Monitoring \& Logging:} Error Tracking und Performance Monitoring aktivieren
    \item \textbf{Warenkorb-Persistierung:} In Datenbank speichern für Multi-Device Unterstützung
    \item \textbf{Admin-Dashboard:} Grafiken für Sales-Analytics, Top-Produkte, Kundenanalyse
\end{enumerate}

\subsection{Mittel-Fristig (3-6 Monate)}
\label{subsec:mittelfristig}

Für die nächsten 3-6 Monate werden folgende Erweiterungen empfohlen:

\begin{enumerate}
    \item \textbf{Skalierung:} Migration zu PostgreSQL + Redis für höhere Concurrency
    \item \textbf{Search:} Elasticsearch für Volltextsuche und Faceted Navigation
    \item \textbf{API:} RESTful API für Mobile-Clients und Drittanbieter-Integration
    \item \textbf{Inventory:} Echtzeit-Lagerbestandsverwaltung mit Low-Stock Alerts
    \item \textbf{Shipping:} DHL/UPS/FedEx API-Integration für automatische Versand-Labels
    \item \textbf{Reviews:} Benutzer-Reviews und Ratings für Produkte
\end{enumerate}

\subsection{Lang-Fristig (6+ Monate)}
\label{subsec:langfristig}

Für die langfristige Entwicklung:

\begin{enumerate}
    \item \textbf{AI/ML:} Personalisierte Produktempfehlungen mit Machine Learning
    \item \textbf{Cloud:} Migration zu Cloud-Infrastructure (AWS/Azure) für globale Verfügbarkeit
    \item \textbf{Microservices:} Evtl. Aufspaltung in Microservices (Auth, Cart, Checkout, Admin)
    \item \textbf{Mobile Apps:} Native iOS/Android Apps basierend auf API
    \item \textbf{Marketplace:} Multi-Vendor Platform für externe Seller
\end{enumerate}

Die modulare Architektur ermöglicht diese Erweiterungen ohne Major Refactoring.

\section{Reflexion des Lernprozesses}
\label{sec:lernprozess}

\subsection{Was war wertvoll}
\label{subsec:wertvoll}

Dieser Entwicklungsprozess bot wertvolle Lernpunkte über mehrere Web-Engineering Kernbereiche:

\paragraph{Anforderungsanalyse \& Design:}

Das strukturierte Erarbeiten von Anforderungen (Zielgruppe → Use Cases → Funktionen) half zu verstehen, wie komplexe Systeme geplant werden. Die Erkenntnis, dass gutes Design die Hälfte der Implementierungsarbeit spart, ist zentral.

\paragraph{Architektur-Entscheidungen:}

Die Wahl von Technologien ist nicht rein technisch, sondern auch strategisch. Das Verständnis für Tradeoffs (SQLite vs. PostgreSQL, Session vs. Datenbank, Monolith vs. Microservices) ist wichtig für gute Engineering-Entscheidungen.

\paragraph{Sicherheit \& Compliance:}

Von theoretischem Wissen über DSGVO zu praktischer Implementierung (Audit-Logs, Datenexport, Löschung) ist ein großer Schritt. Das Projekt hat gezeigt, dass Compliance nicht Overhead ist, sondern Best Practice.

\paragraph{End-to-End Entwicklung:}

Ein vollständiges Projekt vom Konzept bis zum funktionsfähigen MVP zu entwickeln ist anders als einzelne Coding-Übungen. Die vielen Abhängigkeiten und Integrationen erfordern holistic Denken.

\subsection{Besondere Herausforderungen}
\label{subsec:spez_herausforderungen}

Mehrere Aspekte waren besonders herausfordernd:

\begin{itemize}
    \item \textbf{Zahlungsintegration:} Externe APIs mit eigenen Komplexitäten zu integrieren
    \item \textbf{DSGVO-Compliance:} Rechtliche Anforderungen in Code umzusetzen
    \item \textbf{Skalierungs-Denken:} MVP zu bauen, aber bereits für zukünftige Skalierung zu entwerfen
    \item \textbf{Priorisierung:} Mit begrenzter Zeit Features zu priorisieren
\end{itemize}

Jede Herausforderung bot aber auch tieferes Verständnis.

\subsection{Was würde ich anders machen}
\label{subsec:anders}

Reflektiv hätte ich folgendes anders gemacht:

\begin{enumerate}
    \item \textbf{Früher automatisierte Tests:} Statt am Ende, von Anfang an (TDD)
    \item \textbf{Dokumentation parallel:} Statt erst am Ende (Living Documentation)
    \item \textbf{Mehr externe Input:} Code Reviews von anderen Entwicklern würden Qualität erhöhen
    \item \textbf{Prototyping:} Schnellere Prototypen vor finaler Implementierung
    \item \textbf{Buffer:} Mehr Time Buffer für unerwartete Komplexität
\end{enumerate}

Aber insgesamt bin ich mit dem Ergebnis und dem Lernprozess zufrieden.

\section{Finales Fazit}
\label{sec:finales_fazit}

Dieses Projektbericht dokumentiert die erfolgreiche Konzeption und Implementierung eines funktionsfähigen Onlineshops mit modernen Web-Technologies und Best Practices. Das Projekt demonstriert:

\begin{itemize}
    \item \textbf{Praktisches Web-Engineering Know-How:} Von Anforderungsanalyse über Architektur zu Implementierung
    \item \textbf{Sicherheits- und Compliance-Bewusstsein:} DSGVO, PCI-DSS, XSS/CSRF-Protection
    \item \textbf{Problemlösungs-Kompetenz:} Herausforderungen erkennen und pragmatische Lösungen entwickeln
    \item \textbf{Qualitätsorientierung:} Trotz MVP-Ansatz auf hohe Qualität achten
    \item \textbf{Dokumentation \& Reflexion:} Arbeit nachvollziehbar und lernbar machen
\end{itemize}

Der Shop ist nicht nur ein funktionierendes Proof-of-Concept, sondern eine solide Basis für weitere Entwicklung. Die modulare Architektur, gute Code-Qualität und Dokumentation ermöglichen Wartung und Erweiterung durch andere Entwickler.

Vor allem hat dieses Projekt gezeigt, dass durchdachte Planung und systematisches Vorgehen bei komplexen Systemen essentiell sind. Technology ist nur ein Werkzeug — gutes Design ist die Kunst.

\subsection{Abschlussworte}
\label{subsec:abschluss}

Das Projekt war eine bereichernde Erfahrung, die theoretisches Wissen in praktische Fähigkeiten transformiert hat. Die Kombination aus

\begin{itemize}
    \item solider Planung (Anforderungen, Architektur, Design)
    \item pragmatischer Implementierung (Agile, Iterative)
    \item kritischer Reflexion (Lernpunkte, Verbesserungen)
\end{itemize}

bildet die Basis für professionelle Softwareentwicklung. Dieses Projekt ist ein starker Anfang für eine Karriere im Web-Engineering.

\section{Reflexion des Lernprozesses}
\label{sec:lernprozess}

% Platzhalter für Inhalt

% ============================================================================
% LITERATURVERZEICHNIS
% ============================================================================
\begin{thebibliography}{99}

\bibitem{Prüfungsleitfaden} IU Internationale Hochschule GmbH (2024). Prüfungsleitfaden zur Erstellung eines Projektberichts.

\bibitem{Aufgabenstellung} IU Internationale Hochschule GmbH (2024). Aufgabenstellung 2: Konzeption und Umsetzung eines einfachen Onlineshops.

\bibitem{Flask} Ronacher, A. (2024). \textit{Flask Dokumentation}. Verfügbar unter: https://flask.palletsprojects.com/

\bibitem{Python} Python Software Foundation (2024). \textit{Python Dokumentation}. Verfügbar unter: https://docs.python.org/

\bibitem{SQLite} SQLite Consortium (2024). \textit{SQLite Dokumentation}. Verfügbar unter: https://www.sqlite.org/docs.html

\bibitem{DSGVO} Europäisches Parlament und Rat (2016). Verordnung (EU) 2016/679 des Europäischen Parlaments und des Rates vom 27. April 2016 zum Schutz natürlicher Personen bei der Verarbeitung personenbezogener Daten (Datenschutz-Grundverordnung).

\bibitem{PCI-DSS} Payment Card Industry Security Standards Council (2024). \textit{PCI Data Security Standard}. Verfügbar unter: https://www.pcisecuritystandards.org/

\bibitem{PSD2} Europäisches Parlament und Rat (2015). Richtlinie (EU) 2015/2366 vom 25. November 2015 über Zahlungsdienste im Binnenmarkt.

\bibitem{OWASP} OWASP (2024). \textit{OWASP Top 10 – Web Application Security Risks}. Verfügbar unter: https://owasp.org/www-project-top-ten/

% Weitere Literaturangaben können hier ergänzt werden

\end{thebibliography}

% ============================================================================
% ANHANG
% ============================================================================
\appendix

\chapter{Tabellen}
\label{chap:tabellen}

\section{Benutzerrollen und Berechtigungen}
\label{sec:tab:benutzerrollen}

\begin{table}[H]
\centering
\caption{Benutzerrollen mit Authentifizierung und Berechtigungen}
\label{tab:benutzerrollen}
\begin{tabular}{|l|l|p{8cm}|}
\hline
\textbf{Rolle} & \textbf{Authentifizierung} & \textbf{Berechtigungen} \\
\hline
User (Kunde) & E-Mail + Passwort & Browsen, in den Warenkorb legen, Bestellungen aufgeben, Bestellungshistorie einsehen, DSGVO-Rechte nutzen \\
\hline
Admin & E-Mail + Passwort + Admin-PIN & Alle User-Funktionen + Produkte verwalten, Bestellungen überwachen, Benutzer-Daten exportieren/löschen \\
\hline
\end{tabular}
\end{table}

\section{Nicht-funktionale Anforderungen}
\label{sec:tab:nfa}

\begin{table}[H]
\centering
\caption{Nicht-funktionale Anforderungen und deren Zielvorgaben}
\label{tab:nfa}
\begin{tabular}{|l|p{5cm}|l|}
\hline
\textbf{Anforderung} & \textbf{Beschreibung} & \textbf{Zielwert} \\
\hline
Verfügbarkeit & Uptime des Systems & 99\% \\
\hline
Ladezeit & Seitenladung in Browser & < 2 Sekunden \\
\hline
Skalierbarkeit & Gleichzeitige User & Min. 50 Benutzer \\
\hline
Datenschutz & DSGVO-Konformität & 100\% \\
\hline
\end{tabular}
\end{table}

\section{DSGVO-Anforderungen}
\label{sec:tab:dsgvo}

\begin{table}[H]
\centering
\caption{DSGVO-Artikel und ihre Implementierung im Shop}
\label{tab:dsgvo}
\begin{tabular}{|l|p{10cm}|}
\hline
\textbf{Artikel} & \textbf{Implementierung} \\
\hline
Art. 15 (Auskunftsrecht) & Benutzer können ihre persönlichen Daten exportieren \\
\hline
Art. 17 (Recht auf Vergessenwerden) & Benutzer können ihr Konto und alle zugehörigen Daten löschen \\
\hline
Art. 18 (Einschränkung der Verarbeitung) & Audit-Logs dokumentieren alle Datenoperationen \\
\hline
Art. 21 (Widerspruchsrecht) & Benutzer können Newsletter- und Marketing-Einwilligungen widerrufen \\
\hline
\end{tabular}
\end{table}

\section{Authentifizierungsmechanismen}
\label{sec:tab:auth}

\begin{table}[H]
\centering
\caption{Sicherheitsmechanismen für die Authentifizierung}
\label{tab:auth}
\begin{tabular}{|l|p{10cm}|}
\hline
\textbf{Mechanismus} & \textbf{Implementierung} \\
\hline
Passwort-Hashing & PBKDF2 via Werkzeug, 150.000+ Iterationen, eindeutige Salt \\
\hline
Session-Management & Flask-Sessions mit signiertem Secret Key (secure cookies) \\
\hline
Admin-PIN & Zusätzliche PIN für Admin-Registrierung (deterrent) \\
\hline
\end{tabular}
\end{table}

\section{Datenmodell: Tabellenstruktur}
\label{sec:tab:datenmodell}

\begin{table}[H]
\centering
\caption{USER Tabelle - Benutzerdaten}
\label{tab:user}
\begin{tabular}{|l|l|l|}
\hline
\textbf{Spalte} & \textbf{Datentyp} & \textbf{Beschreibung} \\
\hline
id & INTEGER & Primary Key, Auto-Increment \\
\hline
email & VARCHAR(255) & Unique, eindeutige E-Mail-Adresse des Benutzers \\
\hline
password & VARCHAR(255) & Gehashtes Passwort (PBKDF2) \\
\hline
name & VARCHAR(255) & Voller Name des Benutzers \\
\hline
role & VARCHAR(50) & 'user' oder 'admin' \\
\hline
created\_at & TIMESTAMP & Zeitstempel der Registrierung \\
\hline
\end{tabular}
\end{table}

\begin{table}[H]
\centering
\caption{PRODUCT Tabelle - Produktkatalog}
\label{tab:product}
\begin{tabular}{|l|l|l|}
\hline
\textbf{Spalte} & \textbf{Datentyp} & \textbf{Beschreibung} \\
\hline
id & INTEGER & Primary Key, Auto-Increment \\
\hline
name & VARCHAR(255) & Produktname \\
\hline
description & TEXT & Detaillierte Produktbeschreibung \\
\hline
price & DECIMAL(10,2) & Produktpreis in EUR \\
\hline
stock & INTEGER & Verfügbare Menge im Lager \\
\hline
images & JSON & Array von Bildpfaden \\
\hline
created\_at & TIMESTAMP & Zeitstempel der Erstellung \\
\hline
\end{tabular}
\end{table}

\begin{table}[H]
\centering
\caption{ORDER Tabelle - Bestellungen}
\label{tab:order}
\begin{tabular}{|l|l|l|}
\hline
\textbf{Spalte} & \textbf{Datentyp} & \textbf{Beschreibung} \\
\hline
id & INTEGER & Primary Key, Auto-Increment \\
\hline
user\_id & INTEGER & Foreign Key zur USER Tabelle \\
\hline
total\_price & DECIMAL(10,2) & Gesamtbestellwert \\
\hline
status & VARCHAR(50) & \texttt{pending, paid, in\_bearbeitung, versendet, zugestellt} \\
\hline
order\_items & JSON & Array der bestellten Produkte mit Mengen \\
\hline
payment\_method & VARCHAR(50) & 'stripe' oder 'paypal' \\
\hline
created\_at & TIMESTAMP & Bestellzeitpunkt \\
\hline
\end{tabular}
\end{table}

\begin{table}[H]
\centering
\caption{USER\_CONSENT Tabelle - DSGVO Einwilligungen}
\label{tab:consent}
\begin{tabular}{|l|l|l|}
\hline
\textbf{Spalte} & \textbf{Datentyp} & \textbf{Beschreibung} \\
\hline
id & INTEGER & Primary Key, Auto-Increment \\
\hline
user\_id & INTEGER & Foreign Key zur USER Tabelle \\
\hline
consent\_type & VARCHAR(100) & \texttt{privacy\_policy, terms, marketing, analytics} \\
\hline
given & BOOLEAN & True wenn Einwilligung gegeben \\
\hline
given\_at & TIMESTAMP & Zeitpunkt der Einwilligung \\
\hline
\end{tabular}
\end{table}

\begin{table}[H]
\centering
\caption{AUDIT\_LOG Tabelle - Protokollierung aller Operationen}
\label{tab:audit}
\begin{tabular}{|l|l|l|}
\hline
\textbf{Spalte} & \textbf{Datentyp} & \textbf{Beschreibung} \\
\hline
id & INTEGER & Primary Key, Auto-Increment \\
\hline
user\_id & INTEGER & Betroffener Benutzer (optional) \\
\hline
event\_type & VARCHAR(100) & \texttt{USER\_REGISTRATION, USER\_LOGIN, ORDER\_CREATED, DATA\_EXPORT, USER\_DELETED} \\
\hline
timestamp & TIMESTAMP & Zeitpunkt der Operation \\
\hline
ip\_address & VARCHAR(50) & IP-Adresse des Benutzers \\
\hline
details & JSON & Zusätzliche Details der Operation \\
\hline
\end{tabular}
\end{table}

\section{Python Dependencies}
\label{sec:tab:dependencies}

\begin{table}[H]
\centering
\caption{Externe Python-Packages und ihre Verwendung}
\label{tab:dependencies}
\begin{tabular}{|l|l|p{7cm}|}
\hline
\textbf{Package} & \textbf{Version} & \textbf{Verwendungszweck} \\
\hline
Flask & \texttt{\textasciicircum 2.0} & Web-Framework und Routing \\
\hline
pandas & \texttt{\textasciicircum 1.3} & CSV-Datenmanipulation und Analyse \\
\hline
sqlite3 & Built-in & SQL-Datenbank (enthalten in Python) \\
\hline
pytest & Latest & Unit- und Integrationstests \\
\hline
flask-restful & Latest & REST-API-Unterstützung \\
\hline
Werkzeug & Dep. von Flask & Sicherheitsfunktionen (Passwort-Hashing, CSRF) \\
\hline
Jinja2 & Dep. von Flask & Template-Engine für HTML \\
\hline
stripe & Latest & Stripe Payment Gateway SDK \\
\hline
requests & Latest & HTTP-Client für PayPal API \\
\hline
python-dotenv & Latest & Umgebungsvariablen aus \texttt{.env} laden \\
\hline
\end{tabular}
\end{table}

\section{Hybrid-Backend Architektur}
\label{sec:tab:hybrid}

\begin{table}[H]
\centering
\caption{Hybrid-Backend: Fallback-Mechanismus}
\label{tab:hybrid}
\begin{tabular}{|l|l|l|}
\hline
\textbf{Priorität} & \textbf{Speicher-Typ} & \textbf{Beschreibung} \\
\hline
1 & SQLite (Primär) & Native SQL-Datenbank für hohe Performance und Zuverlässigkeit \\
\hline
2 & CSV (Fallback) & Dateisystem-basierte CSV-Dateien bei SQLite-Ausfällen \\
\hline
\end{tabular}
\end{table}

\section{Dateistruktur}
\label{sec:tab:dateien}

\begin{table}[H]
\centering
\caption{Wichtige Projektdateien und deren Zweck}
\label{tab:dateien}
\begin{tabular}{|l|l|}
\hline
\textbf{Datei/Ordner} & \textbf{Zweck} \\
\hline
\texttt{src/app.py} & Haupt-Flask-Anwendung mit allen Routes \\
\hline
\texttt{src/config.py} & Konfiguration und Umgebungsvariablen \\
\hline
\texttt{src/storage/} & Datenbank-Backends (SQLite, CSV, Hybrid) \\
\hline
\texttt{src/services/checkout.py} & Checkout und Zahlungsintegration \\
\hline
\texttt{src/utils/logging\_service.py} & DSGVO-Audit-Logging \\
\hline
\texttt{src/templates/} & Jinja2 HTML-Templates \\
\hline
\texttt{data/} & Laufzeit-Daten (CSV-Dateien, SQLite-DB, Logs) \\
\hline
\end{tabular}
\end{table}

\section{Implementierungs-Phasen}
\label{sec:tab:phasen}

\begin{table}[H]
\centering
\caption{8-Phasen Entwicklungsplan mit Meilensteinen}
\label{tab:phasen}
\begin{tabular}{|l|l|l|}
\hline
\textbf{Phase} & \textbf{Dauer} & \textbf{Meilenstein} \\
\hline
Phase 1 & Woche 1 & Projektsetup, Umgebungskonfiguration, Git-Repository \\
\hline
Phase 2 & Woche 2 & Datenmodell und Datenbank-Schemas definiert \\
\hline
Phase 3 & Woche 3-4 & Benutzerverwaltung (Registrierung, Login) \\
\hline
Phase 4 & Woche 4-5 & Produktkatalog und Warenkorb \\
\hline
Phase 5 & Woche 5-6 & Checkout und Zahlungsintegration (Stripe/PayPal) \\
\hline
Phase 6 & Woche 6-7 & DSGVO-Features und Audit-Logging \\
\hline
Phase 7 & Woche 7-8 & Testing (Manual E2E, Unit Tests, Security) \\
\hline
Phase 8 & Woche 8 & Dokumentation, MVP-Finalisierung \\
\hline
\end{tabular}
\end{table}

\section{Browser-Kompatibilität}
\label{sec:tab:browser}

\begin{table}[H]
\centering
\caption{Getestete Browser und ihre Kompatibilität}
\label{tab:browser}
\begin{tabular}{|l|l|l|l|}
\hline
\textbf{Browser} & \textbf{Version} & \textbf{OS} & \textbf{Status} \\
\hline
Chrome & Aktuell & Win/Mac/Linux & ✓ Voll kompatibel \\
\hline
Firefox & Aktuell & Win/Mac/Linux & ✓ Voll kompatibel \\
\hline
Safari & Aktuell & Mac/iOS & ✓ Voll kompatibel \\
\hline
Edge & Aktuell & Win & ✓ Voll kompatibel \\
\hline
Mobile Chrome & Aktuell & Android & ✓ Responsive Design \\
\hline
Mobile Safari & Aktuell & iOS & ✓ Responsive Design \\
\hline
\end{tabular}
\end{table}

\section{Erfolgs- und Lernfaktoren}
\label{sec:tab:erfolg}

\begin{table}[H]
\centering
\caption{4 Erfolgsfaktoren des Projekts}
\label{tab:erfolg}
\begin{tabular}{|l|p{10cm}|}
\hline
\textbf{Erfolgsfaktor} & \textbf{Beschreibung} \\
\hline
Hybrid-Backend Innovation & Die intelligente Kombination von SQLite und CSV ermöglichte Zuverlässigkeit mit einfacher Wartbarkeit \\
\hline
Fokus auf Benutzer-Sicherheit & DSGVO-Compliance und Sicherheitsmaßnahmen (PBKDF2, CSRF, XSS-Schutz) waren von Anfang an integriert \\
\hline
Manuelle E2E-Testing Strategie & Umfassendes manuelles Testen über 50+ Test-Cases sicherte Qualität ohne Komplexität \\
\hline
Schlanke Architektur & Minimale Dependencies (nur Flask, pandas, Stripe/PayPal) ermöglichten schnelle Entwicklung und einfache Deployment \\
\hline
\end{tabular}
\end{table}

\section{Zielererreichung}
\label{sec:tab:ziele}

\begin{table}[H]
\centering
\caption{Erreichte Ziele vs. Aufgabenstellung}
\label{tab:ziele}
\begin{tabular}{|l|l|l|}
\hline
\textbf{Anforderung} & \textbf{Status} & \textbf{Implementierung} \\
\hline
Benutzerregistrierung und Login & ✓ & Vollständig mit PBKDF2 Hashing \\
\hline
Produktkatalog & ✓ & Komplett mit Bildern und Kategorien \\
\hline
Warenkorb-Verwaltung & ✓ & Session-basiert, AJAX-enabled \\
\hline
Zahlungsintegration & ✓ & Stripe und PayPal delegiert \\
\hline
Bestellverwaltung & ✓ & Vollständig mit Status-Tracking \\
\hline
Admin-Oberfläche & ✓ & Produkt- und Bestell-Management \\
\hline
DSGVO-Compliance & ✓ & Datenexport, Löschung, Audit-Logging \\
\hline
Sicherheit & ✓ & CSRF, XSS, SQLi Schutz \\
\hline
Testing & ✓ & 50+ E2E Test Cases, Unit Tests \\
\hline
\end{tabular}
\end{table}

\chapter{Anhang: Projektdateien und Dokumentation}
\label{chap:anhang}

\section{Projektstruktur}
\label{sec:projektstruktur}

% Platzhalter für Inhalt

\section{Konfigurationsdateien}
\label{sec:konfiguration}

% Platzhalter für Inhalt

\section{Screenshots und Visualisierungen}
\label{sec:screenshots}

% Platzhalter für Inhalt

\section{Testdaten und Beispiele}
\label{sec:testdaten}

% Platzhalter für Inhalt

\section{Setup- und Installationsanleitung}
\label{sec:setup}

% Platzhalter für Inhalt

\end{document}
